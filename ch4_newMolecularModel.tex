
\chapter{A new molecular algorithm for {\sc Satisfiability}}

%<Paragraph> Introduce 
This chapter introduces a new molecular algorithm for {\sc Satisfiability}.  The Distribution algorithm distributes literals from a CNF instance into a set of non-conflicting witnesses.

\section{Distribution algorithm for {\sc Satisfiability}}

%	<Paragraph> Introduce Distribution algorithm	
The Distribution algorithm initially consists of single literal witnesses from a single clause.  Literals from each clause get distributed into non-conflicting witness candidates.  A witness candidate conflicts when it contains both $x_i$ and $\neg x_i$.  In this case the configuration gets removed from the set of non-conflicting witnesses.  

	\subsection{Description of the Distribution algorithm}
		
%		<Paragraph> Describe preconditions	
%		<Paragraph> Describe setup

The {\sc Distribution Algorithm} (Algorithm 4.1.1) starts with the literal assignments of a clause.  Evaluation of subsequent clauses extends the solution space using the {\sc Insert Literal} subroutine (Algorithm 4.1.2).  During each insertion, the literal gets inserted into a potential solution vector.  Table \ref{distributionInsertTable} lists the five possibilities for literal assignment.


\begin{figure}[htbp]
	\renewcommand{\figurename}{Algorithm}
	\renewcommand{\thepseudocode}{\ref{distributionAlgorithm}}
	
	\begin{center}

	\begin{pseudocode}[shadowbox]{Distribution Algorithm}{\phi}
	
	T_S \GETS \{ \texttt{S} \} \\
	
	\FOREACH \text{clause } C \text{ in } \phi \DO
		\BEGIN 
		
			T_C \GETS \emptyset \\
			\FOREACH \text{literal } v \text{ in } C \DO
				\BEGIN
					[T_V, T_S] \GETS \text{split}(T_S) \\
					T_N \GETS \text{extract}(T_V, v_{\texttt{N}})\\
					T_P \GETS \text{extract}(T_V, v_{\texttt{P}})\\
					T_V \GETS \text{append}(T_V, v_{\texttt{P}})\\
					T_C \GETS \text{mix}(T_C, T_P, T_V)\\
				\END\\
				T_S \GETS \emptyset\\
				T_S \GETS \text{mix}(T_S, T_C)\\
				T_S \GETS \text{purify}(T_S)\\
		\END\\
	\RETURN{ \text{detect}(T_S)}
	\end{pseudocode}

\caption{The {\sc Distribution Algorithm} constructs a set of witnesses for a CNF instance $\phi$ by clause evaluation.  The literals of each clause $C$ get distributed into the clause witness tube $T_C$.  These witnesses contained in $T_C$ continue to witness the evaluated portion of $\phi$ in the solution witness tube $T_S$.  Note that the algorithm discards the tube $T_N$ to remove conflicting assignments. }
\label{distributionAlgorithm}
\end{center}
\end{figure}



\FloatBarrier
	

\begin{table}[htdp]
\caption{Configurations for the {\sc Insert Literal} subroutine}
\begin{center}
\begin{tabular}{|c|c|c|}
\hline
\textbf{Case} & \textbf{Return configuration} & \textbf{State} \\ \hline 
1	& $\ell \cdot s$ & if $\ell$ is less than all elements in $s$ \\ 
& &  \\ \hline
2	& $s \cdot \ell$ & if $\ell$ is greater than all elements in $s$ \\ 
& &  \\ \hline
3	& $s_1 \cdot \ell \cdot s_2$ & if $\ell$ is between two elements in $s$ \\ 
& &  \\ \hline
4	& $\emptyset$ & if $\ell$ conflicts with $\neg \ell$ in $s$\\
& &  \\ \hline
5	& $s$ & if $\ell$ exists in $s$\\ \hline
\end{tabular}
\end{center}
\label{distributionInsertTable}
\end{table}%

\FloatBarrier

%		<Paragraph> Describe execution
		
During this phase, each literal from a disjunctive clause gets considered, incrementally constructing a partial solution space.  Cases (1), (2), and (3) place a literal $\ell$ into an existing sequence $s$.  Each of these cases represents when the literal $\ell$ get inserted in a non-decreasing sequence.


\begin{figure}[htbp]
\begin{center}

	\begin{pseudocode}{Insert Literal}{T, \ell}
		T_R \GETS \emptyset \\
		
		\IF T = \emptyset \THEN
			\BEGIN
				\text{// Assign the single literal witness for \textbf{case }(0).}\\
				 T_R \GETS \{ (\ell) \}\\
			\\
			\END
		\ELSE
			\BEGIN	
				\FOREACH \text{ Witness candidate } w \text{ in } T \DO
					\BEGIN
						\CASE(1): \ell < w \THEN
							w' \GETS \text{append}(\ell, w)\\
						\CASE(2): \ell > s \THEN
							w' \GETS \text{append}(w, \ell)\\
						\CASE(3): w_1 < \ell < w_2 \THEN
							\BEGIN
							[w_1, w_2] \GETS \text{splice}(w, \ell)\\
							w_1 \GETS \text{append}(w_1, \ell)\\
							w' \GETS \text{append}(w_1, w_2)\\
							\END \\
						\CASE(4): \neg \ell \in w \THEN
							w' \GETS \emptyset \\
						\CASE(5): \ell \in s \THEN
							w' \GETS w\\
						T_R \GETS \text{mix}(T_R, w') \\
					\END \\
			\END \\
		\RETURN{T_R}
	\end{pseudocode}

\caption{{\sc Insert Literal} maintains the literal assignment $\ell$ to a set of witness candidates contained in $T$.  An ordering of literals is maintained for each oligonucleotide $s$.  Cases (1), (2), and (3) insert the literal assignment $\ell$ into a witness configuration.  Case (4) eliminates conflicting assignments those containing positive and negative literal assignments.  Case (5) does not extend the witness contained in $T$ if the literal assignment is redundant.}
\label{insertVariableAlgorithm}
\end{center}
\end{figure}




A literal conflict occurs when both positive and negative assignments of a literal occur in a sequence $s$.  In this case (4), the sequence $s$ gets removed from the set potential solutions.  If the sequence $s$ contains the literal $\ell$, case (5), then the existing sequence $s$ gets returned unmodified.



%Redundant vectors get removed after insertion of the next disjunctive clause.  Any remaining witnesses in the solution space contain non-conflicting literal assignments.  This does not immediately require that each witness to be a complete satisfiable assignment. Satisfiable witnesses remain in a non-empty satisfying solution space.

For example, let us consider the CNF instance
\[
\phi = (x_1 \vee x_2) \wedge (x_1 \vee \neg x_2 \vee x_3).
\]		

The Distribution algorithm begins with the first clause.  We have the set of witnesses

\[
T = \{ [x_1], [x_2] \}.
\] 

Next, literals from the second clause get inserted into the set of non-conflicting witnesses in $T$.
\[
T_C = \emptyset
\]

\noindent First insert $x_1$ into $T$
\[
T_I = \{ [x_1], [x_1, x_2] \}.
\] 

\noindent The witness $[x_1]$ contains $x_1$, and the idempotent witness $[x_1]$ is returned.  The witness $[x_1, x_2]$ requires $x_1$ to satisfy the second clause.

Mix the contents of $T_I$ into the set of clause witnesses $T_C$
\[
T_C = \{ [x_1], [x_1, x_2] \}.
\]

\noindent Next insert $\neg x_2$ into $T$
\[
T_I = \{ [x_1, \neg x_2] \}.
\] 

\noindent The witness $[x_1, \neg x_2]$ requires $\neg x_2$ to satisfy the second clause.  The witness candidate $[x_2, \neg x_2]$ contains a conflict and gets removed from the set of potential witnesses.

Mix the contents of $T_I$ into the set of clause witnesses $T_C$
\[
T_C = \{ [x_1], [x_1, x_2], [x_1, \neg x_2] \}.
\]


Finally insert the literal $x_3$ into $T$

\[
T_I = \{ [x_1, x_3], [x_2, x_3] \}.
\] 

\noindent In this case, the literal $x_3$ is required to satisfy the second clause.  The literal gets inserted into the configurations without redundant or conflicting assignment.

Mix the contents of $T_I$ into the set of clause witnesses $T_C$
\[
T_C = \{ [x_1], [x_1, x_2], [x_1, \neg x_2], [x_1, x_3], [x_2, x_3] \}.
\]

The contents of $T_C$ get assigned to $T$.

\[
T = \{ [x_1], [x_1, x_2], [x_1, \neg x_2], [x_1, x_3], [x_2, x_3] \}.
\]

The set $T$ contains non-conflicting witnesses for the observed clauses.  Once the algorithm terminates, $T$ contains non-conflicting witnesses for the CNF instance $\phi$. 
	
	\subsection{Detailed trace of the Distribution algorithm}

Appendix B lists a detailed execution trace for the Distribution algorithm.


%
%\section{Simulation of distribution algorithm}
%
%\section{Physical construction of distribution algorithm}
%	


