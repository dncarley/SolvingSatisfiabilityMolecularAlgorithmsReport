
\chapter{A new molecular algorithm for {\sc Satisfiability}}

%<Paragraph> Introduce 
This chapter introduces a new molecular algorithm for {\sc Satisfiability}.  The distribution algorithm parses an input CNF expression into growing and self regulated set of possible combinations.
\section{Distribution algorithm for {\sc Satisfiability}}

%	<Paragraph> Introduce Distribution algorithm	
The distribution algorithm parses an input CNF expression into growing and self regulated set of possible combinations.  A possible combination begins with all members of the first clause.  Variables get inserted into an expanding set of valid assignments.  A clause gets eliminated when an assignment contains a conflict.
	\subsection{Description of the Distribution algorithm}
		
%		<Paragraph> Describe preconditions	
%		<Paragraph> Describe setup
		
Initially the algorithm starts with the variable assignments of a clause.  Evaluation of subsequent clauses extends the solution space with the {\sc Insert Variable} subroutine.  During each insertion, the variable gets inserted into a potential solution vector.  Table \ref{distributionInsertTable} lists the four possibilities for variable assignment.

\begin{table}[htdp]
\caption{Configurations for the {\sc Insert Variable} subroutine}
\begin{center}
\begin{tabular}{|c|c|c|}
\hline
Case & Return state & State \\ \hline 
1	& $v \cdot s$ & if $v$ is less than all elements in $s$ \\ 
& &  \\ \hline
2	& $s \cdot v$ & if $v$ is greater than all elements in $s$ \\ 
& &  \\ \hline
3	& $s_1 \cdot v \cdot s_2$ & if $v$ is between two elements in $s$ \\ 
& &  \\ \hline
4	& $\emptyset$ & if $v$ conflicts with $-v$ in $s$\\
& &  \\ \hline
5	& $s$ & if $v$ exists in $s$\\ \hline
\end{tabular}
\end{center}
\label{distributionInsertTable}
\end{table}%

\FloatBarrier

%		<Paragraph> Describe execution
		
During this phase, each variable from a disjunctive clause gets considered, incrementally constructing a partial solution space.  Cases (1), (2), and (3) place a variable $v$ into an existing sequence $s$.  Each of these cases represents when the variable $v$ get inserted in a non-decreasing sequence.

A variable conflict occurs when both positive and negative assignments of a variable occur in a sequence $s$.  In this case (4), the sequence $s$ gets removed from the set potential solutions.  If the sequence $s$ contains the variable $v$, case (5), then the existing sequence $s$ gets returned unmodified.

Redundant vectors get removed after insertion of the next disjunctive clause.  Any remaining witnesses in the solution space contain non-conflicting variable assignments.  This does not immediately require that each witness to be a complete satisfiable assignment. Satisfiable witnesses remain in a non-empty satisfying solution space.
		
		

%		<Paragraph> Describe Output



Vectors that are of equal magnitude of the number of variables in the problem instance are satisfiable witnesses.  However, there may exist solutions that span only the required satisfiable assignments; that is activate each of the independent clauses with at least one non-conflicting assignment.  This assignment may be the minimum witness for the expression, in the case that the backbone consists of the variables of the maximum witness.

		
	\subsection{Pseudocode for Distribution algorithm}

Algorithms 4.1.1 and 4.1.2 provide pseudocode for the Distribution algorithm.  Appendix B lists a detailed execution trace for the Distribution algorithm.



\begin{figure}[htbp]
	\renewcommand{\figurename}{Algorithm}
	\renewcommand{\thepseudocode}{\ref{distributionAlgorithm}}
	
	\begin{center}

	\begin{pseudocode}[shadowbox]{Distribution Algorithm}{\phi}
	
	T_S \GETS \{ \texttt{S} \} \\
	
	\FOREACH \text{clause } C \text{ in } \phi \DO
		\BEGIN 
		
			T_C \GETS \emptyset \\
			\FOREACH \text{literal } v \text{ in } C \DO
				\BEGIN
					[T_V, T_S] \GETS \text{split}(T_S) \\
					T_N \GETS \text{extract}(T_V, v_{\texttt{N}})\\
					T_P \GETS \text{extract}(T_V, v_{\texttt{P}})\\
					T_V \GETS \text{append}(T_V, v_{\texttt{P}})\\
					T_C \GETS \text{mix}(T_C, T_P, T_V)\\
				\END\\
				T_S \GETS \emptyset\\
				T_S \GETS \text{mix}(T_S, T_C)\\
				T_S \GETS \text{purify}(T_S)\\
		\END\\
	\RETURN{ \text{detect}(T_S)}
	\end{pseudocode}

\caption{The {\sc Distribution Algorithm} constructs a set of witnesses for a CNF instance $\phi$ by clause evaluation.  The literals of each clause $C$ get distributed into the clause witness tube $T_C$.  These witnesses contained in $T_C$ continue to witness the evaluated portion of $\phi$ in the solution witness tube $T_S$.  Note that the algorithm discards the tube $T_N$ to remove conflicting assignments. }
\label{distributionAlgorithm}
\end{center}
\end{figure}




%
%\section{Simulation of distribution algorithm}
%
%\section{Physical construction of distribution algorithm}
%	


