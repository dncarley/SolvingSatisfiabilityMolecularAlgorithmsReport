
\chapter{A new molecular algorithm for {\sc Satisfiability}}

%<Paragraph> Introduce 
This chapter introduces a new molecular algorithm for {\sc Satisfiability}: the Distribution algorithm distributes literals from a CNF instance into a set of non-conflicting witnesses. 

\section{Distribution algorithm for {\sc Satisfiability}}

%	<Paragraph> Introduce Distribution algorithm	
The {\sc Distribution Algorithm} (Algorithm 4.1.1) starts with single literal witnesses from the first clause.  The Distribution algorithm maintains a set of witnesses for each processed clause.  Literal assignments that may be inserted into the witness candidates from the processed clauses we call \emph{non-conflicting witnesses}.  A witness candidate \emph{conflicts} when it contains both $x_i$ and $\neg x_i$ literal assignments.  In this case, the algorithm removes the \emph{conflicting witness} from the set of non-conflicting witnesses.  


\begin{figure}[htbp]
\begin{center}

	\begin{pseudocode}{Distribution Algorithm}{\phi}
	
	\FOREACH \text{clause } C \text{ in } \phi \DO
		\BEGIN 
				T_C \GETS \emptyset \\
			\FOREACH \text{literal } \ell \text{ in } C \DO
				\BEGIN
					T_I \GETS \text{\sc Insert Literal}(T, \ell)\\
					T_C \GETS \text{mix}(T_C, T_I)\\
				\END\\
				T \GETS \text{purify(}T_C\text{)} \\
		\END
	\\
	\RETURN{detect(T)}
	\end{pseudocode}

\caption{{\sc Distribution Algorithm} constructs a set of non-conflicting assignments for a CNF instance.  This algorithm inserts contents of each clause $C_i$ into $T$ with the {\sc Insert Literal} subroutine.}
\label{distributionAlgorithm}
\end{center}
\end{figure}


\FloatBarrier

During execution, the Distribution algorithm maintains an ordering for literal assignments in a set of non-conflicting witness.  The {\sc Insert Literal} subroutine (Algorithm 4.1.2) maintains an ordering to the literals in each ordered witness candidate.

Case 0 initiates an empty tube $T$ with literal assignments from the first clause.  We have the tube

\begin{align*}
T_0 &= \text{{\sc Insert Literal}}(T, x_1)\\
    &= \{ (x_1) \}. \\
\end{align*}	

Let us consider an example to demonstrate the cases for {\sc Insert Literal}.  Let $w = (x_2, \neg x_4, x_5)$ be a partial ordered witness contained in a tube
\[
T = \{ (x_2, \neg x_4, x_5) \}
\]

Case 1:
\begin{align*}
T_1 &= \text{{\sc Insert Literal}}(T, x_1)\\
    &= \{ (x_1) \cdot (x_2, \neg x_4, x_5) \} \\
    &= \{ (x_1, x_2, \neg x_4, x_5) \} 
\end{align*}	

Case 2:
\begin{align*}
T_2 &= \text{{\sc Insert Literal}}(T, x_6)\\
    &= \{ (x_2, \neg x_4, x_5) \cdot (x_6) \} \\
    &= \{ (x_2, \neg x_4, x_5, x_6) \}
\end{align*}	

Case 3:
\begin{align*}
T_3 &= \text{{\sc Insert Literal}}(T, x_3)\\
    &= \{ (x_2) \cdot (x_3) \cdot(\neg x_4, x_5) \} \\
	&= \{ (x_2, x_3, \neg x_4, x_5) \}
\end{align*}	

Case 4:
\begin{align*}
T_4 &= \text{{\sc Insert Literal}}(T, \neg x_2)\\
    &= \{ (x_2, \neg x_4, x_5) \} \\
	&= \{  \}
\end{align*}

Case 5:
\begin{align*}
T_5 &= \text{{\sc Insert Literal}}(T, x_2)\\
    &= \{ (x_2, \neg x_4, x_5) \} \\
\end{align*}

In Case 1, the literal $x_1$ variable precedes the first literal in the ordered witness candidate $w$.  In Case 2 the literal $x_6$ variable succeeds the last literal of the ordered witness candidate $w$.

Case 3 inserts a literal $x_3$ into $w$.  Because the literal $x_3$ variable occurs between the first and last ordered literals, the ordered witness candidate $w$ must be split to incorporate $(x_3)$.  We split $w$ into two components
\begin{align*}
w_1 &= (x_2) \\
w_2 &= (\neg x_4, x_5),
\end{align*}

\noindent and form the completed witness candiate by appending the literal $x_3$ to $w_1$
\[
w_1 = (x_2, x_3).
\]
\noindent Finally form the ordered non-conflicting witness $w$ by appending $w_2$ to $w_1$
\[
w = (x_2, x_3, \neg x_4, x_5).
\]

Case 4 eliminates the conflicting witness; both positive ($x_2$) and negative ($\neg x_2$) literals appear for the same variable in the witness candidate $w$.

In Case 5, the witness $w$ remains unchanged since the literal $x_2$ exists in the witness candidate $w$.


\begin{figure}[htbp]
	\renewcommand{\figurename}{Algorithm}
	\renewcommand{\thepseudocode}{\ref{insertVariableAlgorithm}}
	
	\begin{center}

	\begin{pseudocode}[shadowbox]{Insert Literal}{T, \ell}
		T_R \GETS \emptyset \\
		
		\IF T = \emptyset \THEN
			\BEGIN
				\text{// Assign the single literal witness for \textbf{case }(0).}\\
				 T_R \GETS \{ (\ell) \}\\
			\\
			\END
		\ELSE
			\BEGIN	
				\FOREACH \text{ Witness candidate } w \text{ in } T \DO
					\BEGIN
						\CASE(1): \ell < w \THEN
							w' \GETS \text{append}(\ell, w)\\
						\CASE(2): \ell > s \THEN
							w' \GETS \text{append}(w, \ell)\\
						\CASE(3): w_1 < \ell < w_2 \THEN
							\BEGIN
							[w_1, w_2] \GETS \text{splice}(w, \ell)\\
							w_1 \GETS \text{append}(w_1, \ell)\\
							w' \GETS \text{append}(w_1, w_2)\\
							\END \\
						\CASE(4): \neg \ell \in w \THEN
							w' \GETS \emptyset \\
						\CASE(5): \ell \in s \THEN
							w' \GETS w\\
						T_R \GETS \text{mix}(T_R, w') \\
					\END \\
			\END \\
		\RETURN{T_R}
	\end{pseudocode}

\caption{{\sc Insert Literal} maintains the literal assignment $\ell$ to a set of witness candidates contained in $T$.  An ordering of literals is maintained for each oligonucleotide $s$.  Cases (1), (2), and (3) insert the literal assignment $\ell$ into a witness configuration.  Case (4) eliminates conflicting assignments those containing positive and negative literal assignments.  Case (5) does not extend the witness contained in $T$ if the literal assignment is redundant.}
\label{insertVariableAlgorithm}
\end{center}
\end{figure}


\FloatBarrier

	\subsection{Description of the Distribution algorithm}
		
%		<Paragraph> Describe preconditions	
%		<Paragraph> Describe setup
The {\sc Distribution Algorithm} starts with the literal assignments of a clause.  Evaluation of subsequent clauses extends the witness candidates using the {\sc Insert Literal} subroutine.  Each insertion maintains the set of witness candidates to only those that satisfy the current clause.  Table \ref{distributionInsertTable} lists the six possibilities for literal assignment.
	
\begin{table}[htdp]
\caption{Configurations for the {\sc Insert Literal} subroutine}
\begin{center}
\begin{tabular}{|c|c|c|}
\hline
\textbf{Case} & \textbf{Ordered witness} & \textbf{State} \\ \hline 
0	& $(\ell)$ & if $T=\emptyset$ \\ \hline
1	& $(\ell) \cdot (w)$ & if $v \in \ell$ is less than all literal indicies in $w$ \\ 
& &  \\ \hline
2	& $(w) \cdot (\ell)$ & if $v \in \ell$ is greater than all literal indicies in $w$ \\ 
& &  \\ \hline
3	& $(w_1) \cdot (\ell) \cdot (w_2)$ & if $v \in \ell$ is between two literal indicies in $w$ \\ 
& &  \\ \hline
4	& $\emptyset$ & if $\ell$ conflicts with $\neg \ell$ in $w$\\
& &  \\ \hline
5	& $(w)$ & if $\ell$ exists in $w$\\ \hline
\end{tabular}
\end{center}
\label{distributionInsertTable}
\end{table}%

\FloatBarrier

%		<Paragraph> Describe execution
		
During this phase, each literal from a disjunctive clause incrementally constructs partial non-conflicting witnesses for the input $\phi$.  Case 0 inserts a literal $\ell$ to an unassigned initial tube.  Cases 1, 2, and 3 place a literal $\ell$ into an expanding witness candidate $w$.  Each of these cases maintain an ordering to the expanding set of non-conflicting witness candidates.

A literal conflict occurs when both positive and negative assignments of a literal occur in a witness candidate $w$.  Case 4 removes the conflicting witness candidate $w$ from the set potential solutions.  In Case 5, the witness $w$ remains unmodified; this case occurs when the witness $w$ contains the literal $\ell$.

For example, let us consider the CNF instance
\[
\phi = (x_1 \vee x_2) \wedge (x_1 \vee \neg x_2 \vee x_3).
\]		

The Distribution algorithm begins with the first clause.  We have the set of witnesses
\[
T = \{ (x_1), (x_2) \}.
\] 

Next, insert the literals from the second clause into the set of non-conflicting witnesses in $T$.
\[
T_C = \emptyset
\]

\noindent First, insert $x_1$ into $T$
\[
T_I = \{ (x_1), (x_1, x_2) \}.
\] 

\noindent The witness $(x_1)$ contains $x_1$, and the idempotent witness $(x_1)$ remains unchanged.  The witness $(x_1, x_2)$ requires $x_1$ to satisfy the second clause.

Mix the contents of $T_I$ into the set of clause witnesses $T_C$
\[
T_C = \{ (x_1), (x_1, x_2) \}.
\]

\noindent Next, insert $\neg x_2$ into $T$
\[
T_I = \{ (x_1, \neg x_2) \}.
\] 

\noindent The witness $(x_1, \neg x_2)$ requires $\neg x_2$ to satisfy the second clause.  The witness candidate $(x_2, \neg x_2)$ contains a conflict and gets removed from the set of potential witnesses.

Mix the contents of $T_I$ into the set of clause witnesses $T_C$
\[
T_C = \{ (x_1), (x_1, x_2), (x_1, \neg x_2) \}.
\]

Finally, insert the literal $x_3$ into $T$
\[
T_I = \{ (x_1, x_3), (x_2, x_3) \}.
\] 

\noindent In this case, the literal $x_3$ is required to satisfy the second clause.  The literal gets inserted into the configurations without redundant or conflicting assignment.

Mix the contents of $T_I$ into the set of clause witnesses $T_C$
\[
T_C = \{ (x_1), (x_1, x_2), (x_1, \neg x_2), (x_1, x_3), (x_2, x_3) \}.
\]

Assign the contents of $T_C$ to $T$

\[
T = \{ (x_1), (x_1, x_2), (x_1, \neg x_2), (x_1, x_3), (x_2, x_3) \}.
\]

The set $T$ contains non-conflicting witnesses for the observed clauses.  Once the algorithm terminates, $T$ contains non-conflicting witnesses for the CNF instance $\phi$. 
	
	\subsection{Detailed trace of the Distribution algorithm}

Appendix B lists a detailed execution trace for the Distribution algorithm.

%
%\section{Simulation of distribution algorithm}
%
%\section{Physical construction of distribution algorithm}
%	
