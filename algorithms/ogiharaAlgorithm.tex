
\begin{figure}[htbp]
	\renewcommand{\figurename}{Algorithm}
	\renewcommand{\thepseudocode}{\ref{ogiharaRayAlgorithm}}
	
	\begin{center}

	\begin{pseudocode}[shadowbox]{Ogihara and Ray's Algorithm}{\phi}
	
	\text{// Input $\phi$ consists of $n$ variables.}\\
	\text{// Each clause $C$ contains ordered literals $(a,b,c)$.}\\
	\text{// Extract literals using the condensed notation $v_P$, where: }\\
	\text{// \hspace{1em} $v_P$ matches the literal, and $v_N$ matches the negated literal.}\\
	\\
	T \GETS \{ \texttt{STT}, \texttt{STF}, \texttt{SFT},  \texttt{SFF}\} \\
	
	\FOR v \GETS 3 \text{ to } n \DO
		\BEGIN
		[T_P, T_N] \GETS \text{split}(T)\\
	
		\FOREACH \text{clause } C \text{ in } \phi \DO
			\BEGIN
				(a, b, c) \GETS C\\
				\IF v_{\texttt{T}} = c  \THEN
					\BEGIN
						T_{P1} \GETS \text{extract}(T_N, a_P)\\
						T_{N1} \GETS \text{extract}(T_N, a_N)\\				
						T_{P2} \GETS \text{extract}(T_{N1}, b_P)\\
						T_{N} \GETS \text{mix}(T_{P1}, T_{P2})\\
						T_{N} \GETS \text{purify(}T_{N}\text{)}					
					\END \\  
				\IF v_{\texttt{F}} = c \THEN
					\BEGIN
						T_{P1} \GETS \text{extract}(T_P, a_P)\\
						T_{N1} \GETS \text{extract}(T_P, a_N)\\				
						T_{P2} \GETS \text{extract}(T_{N1}, b_P)\\
						T_{P} \GETS \text{mix}(T_{P1}, T_{P2})\\
						T_{P} \GETS \text{purify(}T_{P}\text{)} 						
					\END\\
			\END\\
			T_P \GETS \text{append}(T_P, v_{\texttt{T}})\\
			T_N \GETS \text{append}(T_N, v_{\texttt{F}})\\
			T \GETS \text{mix}(T_P, T_N)\\
			T \GETS \text{purify(}T\text{)} \\									
		\END\\
	\RETURN{\text{detect}(T)}
	\end{pseudocode}

\caption{{\sc Ogihara and Ray's Algorithm} evaluates each subsequent variable and determines possible assignments.  The possible assignments for the variables $a$ and $b$ get extracted if $c$ matches the current variable $v$.  Effectively pruning only potential solutions.  These potential solutions $T_P$ and $T_N$ get appended with the positive or negative string assignments.  The algorithm continues until each variable gets evaluated.  The remaining space $T$ contains all solutions for the CNF instance $\phi$ after the algorithm terminates.}
\label{ogiharaRayAlgorithm}
\end{center}
\end{figure}
