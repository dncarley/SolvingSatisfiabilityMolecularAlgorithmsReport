
\begin{figure}[htbp]
\begin{center}

	\begin{pseudocode}{Ogihara and Ray's Algorithm}{\phi}
	
	n \text{ number of variables in } \phi \\
	\text{Each variable of the reordered clause can be accessed by $v_1$, $v_2$, and $v_3$}\\
	\\
	\text{Reorder variables by most frequent to least frequent literal appearance}\\
	\text{Reorder each clause in increasing literal order}\\
	\\
	T \GETS \{ [+x_1 \cdot +x_2], [+x_1 \cdot -x_2], [-x_1 \cdot +x_2],  [-x_1 \cdot -x_2]\} \\
	
	\FOREACH \text{variable } x_i \text{ in } 3 \leq i \leq n \DO
		\BEGIN
		[T_P, T_N] \GETS \text{split}(T)\\
	
		\FOREACH \text{clause } C \text{ in } \phi \DO
			\BEGIN
				[v_1, v_2, v_3] \GETS C\\
				\IF x_i = v_3  \THEN
					\BEGIN
						T_{P1} \GETS \text{extract}(T_N, v_1)\\
						T_{N1} \GETS \text{extract}(T_N, -v_1)\\				
						T_{P2} \GETS \text{extract}(T_{N1}, v_2)\\
						T_{N} \GETS \text{mix}(T_{P1}, T_{P2})
					\END \\  
				\IF \neg x_i = v_3 \THEN
					\BEGIN
						T_{P1} \GETS \text{extract}(T_P, v_1)\\
						T_{N1} \GETS \text{extract}(T_P, -v_1)\\				
						T_{P2} \GETS \text{extract}(T_{N1}, v_2)\\
						T_{P} \GETS \text{mix}(T_{P1}, T_{P2})
					\END\\
			\END\\
			T_P \GETS \text{append}(T_P, +x_i)\\
			T_N \GETS \text{append}(T_N, -x_i)\\
			T \GETS \text{mix}(T_P, T_N)\\
		\END\\
	\RETURN{\text{detect}(T)}
	\end{pseudocode}

\caption{{\sc Ogihara and Ray's Algorithm} evaluates each subsequent variable and determines possible assignments.  The possible assignments for the variables $v_1$ and $v_2$ get extracted if $v_3$ matches.  Effectively pruning only potential solutions.  These potential solutions $T_P$ and $T_N$ get appended with the positive or negative string assignments.  The algorithm continues until each variable gets evaluated.  The remaining space $T$ contains all solutions for the CNF instance $\phi$ after the algorithm terminates.}
\label{ogiharaRayAlgorithm}
\end{center}
\end{figure}

