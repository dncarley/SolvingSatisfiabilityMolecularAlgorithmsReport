
\begin{figure}[htbp]
\begin{center}

	\begin{pseudocode}{ExtractSat}{\phi}
	
	n \text{ number of variables in } \phi \\
	\\
		T \GETS \text{\sc CombinatorialGenerate}(n)\\
	
		\FOREACH \text{clause } C \text{ in } \phi \DO
			\BEGIN
				T_C \GETS \emptyset \\
			
				\FOREACH \text{literal } \ell \text{ in } C \DO
					\BEGIN 
						T_T \GETS \text{extract}(T, \ell) \\
						T_C \GETS \text{mix}(T_C, T_T)
					\END \\
				T \GETS T_C \\
			\END \\
			
		\IF T = \emptyset
			\THEN \RETURN{\texttt{UNSATISFIABLE}}\\
		\RETURN{\texttt{SATISFIABLE}}
	\end{pseudocode}


\caption{$\text{\sc ExtractSat}(\phi)$ collects satisfying Boolean variables from each clause in $\phi$.  Initially, $\text{\sc ExtractSat}(\phi)$ constructs a combinatorial space $T$ using the subroutine $\text{\sc CombinatorialGenerate}(n)$.  The initial space $T$ contains string configurations representing all potential witness candidates for $\phi$.  The space $T$ gets filtered down to witness all clauses.  These potential solutions are incrementally mixed into the tube $T_C$ for each clause.  }
\label{extractSatAlgorithm}
\end{center}
\end{figure}
