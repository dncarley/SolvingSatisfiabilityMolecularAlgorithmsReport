
\begin{figure}[htbp]
\begin{center}

	\begin{pseudocode}{Insert Literal}{T, \ell}
		T_R \GETS \emptyset \\
		
		\IF T = \emptyset \THEN
			\BEGIN
				\text{// Assign the single literal witness for \textbf{case }(0).}\\
				 T_R \GETS \{ (\ell) \}\\
			\\
			\END
		\ELSE
			\BEGIN	
				\FOREACH \text{ Witness candidate } w \text{ in } T \DO
					\BEGIN
						\CASE(1): \ell < w \THEN
							w' \GETS \text{append}(\ell, w)\\
						\CASE(2): \ell > s \THEN
							w' \GETS \text{append}(w, \ell)\\
						\CASE(3): w_1 < \ell < w_2 \THEN
							\BEGIN
							[w_1, w_2] \GETS \text{splice}(w, \ell)\\
							w_1 \GETS \text{append}(w_1, \ell)\\
							w' \GETS \text{append}(w_1, w_2)\\
							\END \\
						\CASE(4): \neg \ell \in w \THEN
							w' \GETS \emptyset \\
						\CASE(5): \ell \in s \THEN
							w' \GETS w\\
						T_R \GETS \text{mix}(T_R, w') \\
					\END \\
			\END \\
		\RETURN{T_R}
	\end{pseudocode}

\caption{{\sc Insert Literal} maintains the literal assignment $\ell$ to a set of witness candidates contained in $T$.  An ordering of literals is maintained for each oligonucleotide $s$.  Cases (1), (2), and (3) insert the literal assignment $\ell$ into a witness configuration.  Case (4) eliminates conflicting assignments those containing positive and negative literal assignments.  Case (5) does not extend the witness contained in $T$ if the literal assignment is redundant.}
\label{insertVariableAlgorithm}
\end{center}
\end{figure}

