
\begin{figure}[htbp]
	\renewcommand{\figurename}{Algorithm}
	\renewcommand{\thepseudocode}{\ref{distributionAlgorithm}}
	
	\begin{center}

	\begin{pseudocode}[shadowbox]{Distribution Algorithm}{\phi}
	
	T_S \GETS \{ \texttt{S} \} \\
	
	\FOREACH \text{clause } C \text{ in } \phi \DO
		\BEGIN 
		
			T_C \GETS \emptyset \\
			\FOREACH \text{literal } v \text{ in } C \DO
				\BEGIN
					[T_V, T_S] \GETS \text{split}(T_S) \\
					T_N \GETS \text{extract}(T_V, v_{\texttt{N}})\\
					T_P \GETS \text{extract}(T_V, v_{\texttt{P}})\\
					T_V \GETS \text{append}(T_V, v_{\texttt{P}})\\
					T_C \GETS \text{mix}(T_C, T_P, T_V)\\
				\END\\
				T_S \GETS \emptyset\\
				T_S \GETS \text{mix}(T_S, T_C)\\
				T_S \GETS \text{purify}(T_S)\\
		\END\\
	\RETURN{ \text{detect}(T_S)}
	\end{pseudocode}

\caption{The {\sc Distribution Algorithm} constructs a set of witnesses for a CNF instance $\phi$ by clause evaluation.  The literals of each clause $C$ get distributed into the clause witness tube $T_C$.  These witnesses contained in $T_C$ continue to witness the evaluated portion of $\phi$ in the solution witness tube $T_S$.  Note that the algorithm discards the tube $T_N$ to remove conflicting assignments. }
\label{distributionAlgorithm}
\end{center}
\end{figure}

