
\begin{figure}[htbp]
	\renewcommand{\figurename}{Algorithm}
	\renewcommand{\thepseudocode}{\ref{checkSat}}
	
	\begin{center}

	\begin{pseudocode}[shadowbox]{CheckSat}{\phi, B}
	
		\FOREACH \text{clause } C \text{ in } \phi \DO
			\BEGIN
				test\_clause \GETS False \\
			
				\FOREACH \text{literal } \ell \text{ in } C \DO
					\BEGIN
						\IF B \text{ satisfies } \ell 
							\THEN test\_clause \GETS True \\
					\END \\
				\IF test\_clause = False
					\THEN \RETURN{False} \\
			\END \\
		\RETURN{True}
	\end{pseudocode}

\caption{$\text{\sc CheckSat}(\phi, B)$ iterates over each of the clauses $C$ in the CNF instance $\phi$.  The bit-vector $B$ encodes a witness candidate for the CNF instance $\phi$.  The $test\_clause$ variable gets set to $False$, assuming that the clause cannot be satisfied.  If the clause can be satisfied with the input configuration $B$, then the algorithm continues.  If each of the $m$ clauses can be satisfied, then $\text{\sc CheckSat}(\phi, B)$ returns $True$; otherwise the algorithm returns $False$.}
\label{checkSat}
\end{center}
\end{figure}







