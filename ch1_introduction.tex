\chapter{Introduction}

%This chapter provides a a brief introduction to molecular computation.  

Molecular computing uses parallel interactions between genetic molecules, such as DNA or RNA, to perform computational tasks.  We provide an experimental environment for simulating three molecular algorithms.  This environment simulates the execution of random $3$-{\sc Sat} instances.  The $3$-{\sc Sat} instances span discrete clause-variable ratios from 0.2 to 14.0 in increments of 0.2, creating a sweep of {\sc Satisfiability} instances.  This chapter introduces the contents of the project.
%Finally, we provide an introduction to physical gene sequencing techniques for generalized molecular computation.  

\section{Introduction to molecular computation}
	
%<Paragraph> Thesis statement
%				<active sentence> <idea>:{Exp space in polynomial time}
				%	A machine built with exponential space constructs configurations for a NP-complete problem instance in polynomial time.  In this project, we present a simulation environment for solving {\sc Satisfiability} with molecular algorithms. 
				
Molecular interactions test many potential states for discrete states of matter.  We consider genetic encodings as a witnessing mechanism for computational configurations.  Hydrogen bonds form complementary base pairs in DNA and RNA.  Complementary genetic string representations encode data for both storage and matching mechanism.  Molecular computing takes advantage of molecular interactions for general purpose computation.
%%In this project, we consider molecular algorithms for solving {\sc Satisfiability}.				
%		<Paragraph> Introduce algorithms
%				<active sentence> <idea>:{}

	We consider three molecular algorithms for solving {\sc Satisfiability}: Lipton's algorithm \cite{Lipton95usingdna}, Ogihara and Ray's algorithm \cite{Ogihara:1996:BFS:898228, Ogihara97dna-basedparallel} and a new algorithm, introduced here, that we call the `Distribution' algorithm.  Lipton's algorithm enumerates a combinatorial space and gets reduced to satisfiable solutions.  Ogihara and Ray's algorithm constructs a space of potential solutions and eliminates non-satisfiable paths.  The Distribution algorithm constructs a set of non-conflicting states for a satisfiable solution.  Chapters 3 and 4 discuss the implementation of these algorithms.
				
\section{Simulation environment and physical devices}
	
%		<Paragraph> Introduce experiment
%		<Paragraph> Introduce implementation
This project introduces a system for simulating three molecular algorithms for solving {\sc Satisfiability}.  The system provides standard operations for molecular computing that we introduce in Chapter 2.  The system records runtime metrics, including counts of molecular operators, solution memory footprint and execution time.  These metrics lets us analyze algorithm performance for each {\sc Satisfiability} test instance.

%		<Paragraph> Introduce physical machine
Existing fabrication techniques for constructing nanopores \cite{ionTorrent, oxfordNanopore} and micropumps \cite{Liao_Lee_Liu_Hsieh_Luo_2005} provide the technology for sequencing genes.  These gene sequencing technologies, discussed in Chapter 7, provide design considerations for an integrated molecular computing architecture.  In the next chapter, we introduce techniques from nanotechnology, microbiology and theoretical computer science for applied molecular computation.
