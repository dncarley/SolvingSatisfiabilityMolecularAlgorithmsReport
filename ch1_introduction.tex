\chapter{Introduction}

%This chapter provides a a brief introduction to molecular computation.  

Molecular computation uses interactions between genetic molecules, such as DNA or RNA, to perform computational tasks.  We provide an experimental system for simulating three molecular algorithms.  In this chapter we discuss the advantages of molecular computation versus standard computation.  This discussion includes an introduction to the simulation of molecular algorithms.  We conclude the chapter with an overview of the contents of this report.
%Finally, we provide an introduction to physical gene sequencing techniques for generalized molecular computation.  

\section{Introduction to molecular computation}
	
%<Paragraph> Thesis statement
%				<active sentence> <idea>:{Exp space in polynomial time}
				%	A machine built with exponential space constructs configurations for a NP-complete problem instance in polynomial time.  In this project, we present a simulation system for solving {\sc Satisfiability} with molecular algorithms. 
				
\textsf{NP} problems, such as {\sc Satisfiability}, may be verified in polynomial time with the aid of a short (i.e. of length polynomial in the length of the input) proof called a \textit{witness}; \textsf{NP} problems may be solved by checking all possible \textit{witness candidates}.  In a standard computational environment, brute force search checks all witness candidates in exponential time.

Molecular computation requires exponential space in order to represent all witness candidates.  This combinatorial space of witness candidates can be filtered in polynomial time by parallel molecular operators.

A Boolean formula $\phi$ is said to be in Conjunctive Normal Form (CNF) if $\phi$ consists of a conjunctive set of Boolean disjunctive clauses.  (Throughout this report we use CNF instances $\phi$ with $n$ variables, $m$ clauses, and $k$ literals.)  We have, e.g.,
\[
\phi = C_1 \wedge C_2 \wedge \cdots \wedge C_m, 
\]
where each clause $C_i$ contains $k$ disjunctive Boolean literals
\[
C_i = (v_1 \vee v_2 \vee \cdots \vee v_k).
\]

A witness for a {\sc Satisfiability} instance is a Boolean assignment to the variables that makes the formula true.  Such a witness can be represented as a vector $B$ in $\{0,1\}^n$, where $n$ is the number of variables in $\phi$, as follows.  Let $\{v_1, \ldots , v_n\}$ be the variables of $\phi$, for each $i$ in $\{1, \ldots , n\}$, the $i$th element of $B$ is denoted $B_i$ and represents an assignment of true or false to $v_i$.   A witness candidate for a {\sc Satisfiability} instance may be verified in polynomial time with $\text{\sc CheckSat}(\phi, B)$ (Algorithm 1.1.1 below).

\FloatBarrier 


\begin{figure}[htbp]
	\renewcommand{\figurename}{Algorithm}
	\renewcommand{\thepseudocode}{\ref{checkSat}}
	
	\begin{center}

	\begin{pseudocode}[shadowbox]{CheckSat}{\phi, B}
	
		\FOREACH \text{clause } C \text{ in } \phi \DO
			\BEGIN
				test\_clause \GETS False \\
			
				\FOREACH \text{literal } \ell \text{ in } C \DO
					\BEGIN
						\IF B \text{ satisfies } \ell 
							\THEN test\_clause \GETS True \\
					\END \\
				\IF test\_clause = False
					\THEN \RETURN{False} \\
			\END \\
		\RETURN{True}
	\end{pseudocode}

\caption{$\text{\sc CheckSat}(\phi, B)$ iterates over each of the clauses $C$ in the CNF instance $\phi$.  The bit-vector $B$ encodes a witness candidate for the CNF instance $\phi$.  The $test\_clause$ variable gets set to $False$, assuming that the clause cannot be satisfied.  If the clause can be satisfied with the input configuration $B$, then the algorithm continues.  If each of the $m$ clauses can be satisfied, then $\text{\sc CheckSat}(\phi, B)$ returns $True$; otherwise the algorithm returns $False$.}
\label{checkSat}
\end{center}
\end{figure}









\FloatBarrier 

$\text{\sc CheckSat}(\phi, B)$ may be used as a subroutine in a brute force {\sc Satisfiability} solver.  Algorithm 1.1.2 provides pseudocode for a brute force {\sc Satisfiability} solver $\text{\sc BruteSat}(\phi)$.  

\FloatBarrier 



\begin{figure}[htbp]
\begin{center}

\begin{pseudocode}{BruteSat}{\phi}
	n \text{ number of variables in }\phi \\
	t \text{ is bit-vector representing a witness candidate}  \\
\\
	\FOREACH t \in \{0, 1\}^n \DO
		\BEGIN
			\IF \text{\sc CheckSat}(\phi, t) 
				\THEN	\RETURN{\texttt{SATISFIABLE}} \\

		\END \\
	\RETURN{\texttt{UNSATISFIABLE}}
\end{pseudocode}

\caption{$\text{\sc BruteSat}(\phi)$ tests a maximum of $2^n$ Boolean witness candiates, using the $\text{\sc CheckSat}(\phi, B)$ algorithm.  If the test configuration $t$ satisfies the input instance $\phi$, then the algorithm returns \texttt{SATISFIABLE}; otherwise the algorithm returns \texttt{UNSATISFIABLE}.}
\label{bruteSat}
\end{center}
\end{figure}


In this project, we consider molecular algorithms to solve {\sc Satisfiability}.  Molecular algorithms permit parallelism on a massive scale \cite{Adleman:1994:MCS:189441.189442, Lipton95usingdna}.  Molecular operations, such as \textit{append} or \textit{extract}, can perform in parallel on all of the string contents of a test tube \cite{Adleman:1994:MCS:189441.189442, Lipton95usingdna, dnaComputingModels2008}.  In Chapter 3, we explore techniques from combinatorial chemistry to generate combinatorial sets \cite{Lipton95usingdna, furkaBook, dnaComputingModels2008}.


\begin{figure}[htbp]
\begin{center}

	\begin{pseudocode}{ExtractSat}{\phi}
	
	n \text{ number of variables in } \phi \\
	\\
		T \GETS \text{\sc CombinatorialGenerate}(n)\\
	
		\FOREACH \text{clause } C \text{ in } \phi \DO
			\BEGIN
				T_C \GETS \emptyset \\
			
				\FOREACH \text{literal } \ell \text{ in } C \DO
					\BEGIN 
						T_T \GETS \text{extract}(T, \ell) \\
						T_C \GETS \text{mix}(T_C, T_T)
					\END \\
				T \GETS T_C \\
			\END \\
			
		\IF T = \emptyset
			\THEN \RETURN{\texttt{UNSATISFIABLE}}\\
		\RETURN{\texttt{SATISFIABLE}}
	\end{pseudocode}


\caption{$\text{\sc ExtractSat}(\phi)$ collects satisfying Boolean literals from each clause in $\phi$.  Initially, $\text{\sc ExtractSat}(\phi)$ constructs a combinatorial space $T$ using the subroutine $\text{\sc CombinatorialGenerate}(n)$.  The initial space $T$ contains string configurations representing all potential witness candidates for $\phi$.  The space $T$ gets filtered down to witness all clauses.  These potential solutions are incrementally mixed into the tube $T_C$ for each clause.  }
\label{extractSatAlgorithm}
\end{center}
\end{figure}

\FloatBarrier 

Let us consider Algorithm 1.1.3 as a simplified version of Lipton's algorithm \cite{Lipton95usingdna, dnaComputingModels2008}.  The $\text{\sc ExtractSat}(\phi)$ function provides an introductory view of a molecular algorithm.  $\text{\sc ExtractSat}$ differs in how the algorithm validates each candidate.  The brute force algorithm, $\text{\sc BruteSat}$, generates sequentially an exponential number of witness candidates.  On the other hand, exponential witness canidates with $\text{\sc ExtractSat}$ get filtered in parallel.  
				
\section{Simulation of molecular {\sc Satisfiability} solvers}
	
We consider three molecular algorithms for solving {\sc Satisfiability}: Lipton's \cite{Lipton95usingdna}, Ogihara and Ray's \cite{Ogihara:1996:BFS:898228, Ogihara97dna-basedparallel}, and a new algorithm, introduced here, that we call the `Distribution' algorithm.  Lipton's algorithm begins with a combinatorial space of all $n$-bit witness candidates and filters the combinatorial space so that only those that satisfy the input formula remain.  Ogihara and Ray's algorithm constructs a space of witness candidates using heuristic search.  The Distribution algorithm expands a set of witnesses with non-conflicting literals from each clause.  Chapters 3 and 4 discuss the implementation of these algorithms.

This project introduces a {\sc Satisfiability} solver framework for molecular algorithms, which we call `Molecular Simulation'.  This system provides standard operations for molecular computation which we introduce in Chapter 2.  It also records runtime metrics, including counts of molecular operators, memory footprints, and execution times.  These metrics let us analyze the algorithmic performance of each molecular algorithm.

Molecular Simulation automates execution of DIMACS CNF instances.  It measures key properties for a set of randomly generated $3$-{\sc Sat} instances.  The $3$-{\sc Sat} instances span discrete clause-variable ratios from 0.2 to 14.0 in increments of 0.2, creating a sweep of {\sc Satisfiability} instances.  This experimental setup generates {\sc Satisfiability} problem instances with both \texttt{SATISFIABLE} and \texttt{UNSATISFIABLE} configurations.

\section{Report Overview}

In the following chapters, we describe molecular algorithms for solving {\sc Satisfiability}.  We begin Chapter 2 with an introduction to gene sequencing technologies and molecular biology.  We define molecular operations for operating on DNA or RNA.  Next, we introduce {\sc Satisfiability} as a language and as a Boolean circuit.

Chapters 3 and 4 introduce each of the three molecular algorithms for solving {\sc Satisfiability}.  In Chapter 3, we discuss Lipton's \cite{Lipton95usingdna, dnaComputingModels2008} and Ogihara and Ray's \cite{Ogihara:1996:BFS:898228, Ogihara97dna-basedparallel, dnaBasedImplemetation_Yoshida2000} algorithms for {\sc Satisfiability}.  The chapter concludes with a discussion of existing simulation frameworks and physical implementations of these molecular algorithms.  Chapter 4 introduces the Distribution algorithm.

Chapters 5 and 6 discuss the project implementation.  In Chapter 5, we introduce our software, Molecular Simulation, for simulating molecular algorithms.  Chapter 6 describes the experimental workflow for importing {\sc Satisfiability} instances for each of the three molecular algorithms we study.

Chapter 7 provides a discussion of algorithm performance based test results.  Chapter 8 concludes with a summary of contributions of this project and future directions for molecular computation.