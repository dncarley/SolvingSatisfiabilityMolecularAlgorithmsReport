\documentclass[11pt]{report}
\usepackage{geometry}
\geometry{letterpaper}
\usepackage{amssymb}
\usepackage{mathtools}
\usepackage{pseudocode}
\usepackage{fancybox}
\usepackage{graphicx}
\usepackage{epstopdf}
\usepackage{multicol}
\usepackage{placeins}
\usepackage{etoolbox}
\usepackage{url}
\usepackage{hyperref}
\patchcmd{\quote}{\rightmargin}{\leftmargin 15em \rightmargin}{}{}
\DeclareGraphicsRule{.tif}{png}{.png}{`convert #1 `dirname #1`/`basename #1 .tif`.png}

\widowpenalty=300
\clubpenalty=300

\hypersetup{
	colorlinks,%
	citecolor=black,%
	filecolor=black,%
	linkcolor=black,%
	urlcolor=black
}

%%%%%%%%%% http://www.maths.tcd.ie/~dwilkins/LaTeXPrimer/Theorems.html
\newtheorem{theorem}{Theorem}[section]
\newtheorem{lemma}[theorem]{Lemma}
\newtheorem{proposition}[theorem]{Proposition}
\newtheorem{corollary}[theorem]{Corollary}

\newenvironment{proof}[1][Proof]{\begin{trivlist}
\item[\hskip \labelsep {\bfseries #1}]}{\end{trivlist}}
\newenvironment{definition}[1][Definition]{\begin{trivlist}
\item[\hskip \labelsep {\bfseries #1}]}{\end{trivlist}}
\newenvironment{example}[1][Example]{\begin{trivlist}
\item[\hskip \labelsep {\bfseries #1}]}{\end{trivlist}}
\newenvironment{remark}[1][Remark]{\begin{trivlist}
\item[\hskip \labelsep {\bfseries #1}]}{\end{trivlist}}

\newcommand{\qed}{\nobreak \ifvmode \relax \else
	  \ifdim\lastskip<1.5em \hskip-\lastskip
	  \hskip1.5em plus0em minus0.5em \fi \nobreak
	  \vrule height0.75em width0.5em depth0.25em\fi}
%%%%%%%%%% end <http://www.maths.tcd.ie/~dwilkins/LaTeXPrimer/Theorems.html>



\begin{document}

\begin{titlepage}

	\begin{center}
	
		% Upper part of the page		
		{\LARGE Solving {\sc Satisfiability} with Molecular Algorithms}\\[1.5cm]
		by\\
		\vspace{0.5cm}
		{\large David Carley}\\

		\vspace{1cm}
		
{\large Master of Science Project}\\
				\vspace{0.5cm}
Presented to the Faculty of the Graduate School of\\
		\vspace{0.5cm}				
Rochester Institute of Technology\\
		\vspace{0.5cm}
in Partial Fulfillment of the Requirements for the Degree of\\
		\vspace{0.5cm}
{\large Master of Science}
		
		\vspace{1cm}
		
		\begin{minipage}{0.4\textwidth}
			\begin{flushleft} \large

			\end{flushleft}
			\end{minipage}
			\begin{minipage}{0.4\textwidth}
			\begin{flushleft} \large

				\emph{Chair:} \\
				Dr. Christopher Homan
				\vspace{1cm}\\
				\line(1,0){170}\\
				\vspace{0.2cm}

				\emph{Reader:} \\
				Dr. Stanis\l aw Radziszowski
				\vspace{1cm}\\
				\line(1,0){170}\\
				\vspace{0.2cm}
				
				\emph{Observer:} \\
				Dr. Reynold Bailey
				\vspace{1cm}\\
				\line(1,0){170}
				
			\end{flushleft}
		\end{minipage}
	
		\vfill
		
		% Bottom of the page
		{\large October 12, 2012}
	
	\end{center}

\end{titlepage}

\thispagestyle{empty} 
\begin{center}
	\begin{abstract}
Molecular computation uses techniques from molecular biology and combinatorial chemistry to perform generalized computations.  We explore, via simulation, three distinct molecular algorithms for solving {\sc Satisfiability}.  The simulation measures the number of molecular operations for solving {\sc Satisfiability}. The test input consists of a set of random $3$-{\sc Sat} instances distributed over a range of clause-variable ratios ($\alpha = [0.2, 14.0]$). 
	\end{abstract}
\end{center}

\pagenumbering{roman}
\tableofcontents

\clearpage
\pagenumbering{arabic}
\chapter{Introduction}

%This chapter provides a a brief introduction to molecular computation.  

Molecular computing uses genetic information for parallel molecular interactions.  We provide an experimental environment for simulating a sweep of {\sc Satisfiability} instances.  This chapter introduces the contents of the project.
%Finally, we provide an introduction to physical gene sequencing techniques for generalized molecular computation.  

\section{Introduction to molecular computation}
	
%<Paragraph> Thesis statement
%				<active sentence> <idea>:{Exp space in polynomial time}
				%	A machine built with exponential space constructs configurations for a NP-complete problem instance in polynomial time.  In this project, we present a simulation environment for solving {\sc Satisfiability} with molecular algorithms. 
				
Molecular computation harnesses the efficiencies of molecular interactions of genetic information for generalized computation.  %

%%In this project, we consider molecular algorithms for solving {\sc Satisfiability}.				
%		<Paragraph> Introduce algorithms
%				<active sentence> <idea>:{}

	We consider three molecular algorithms for solving {\sc Satisfiability}.  Lipton's algorithm \cite{Lipton95usingdna}, Ogihara and Ray's algorithm \cite{Ogihara:1996:BFS:898228, Ogihara97dna-basedparallel}, and the Distribution algorithm.  Lipton's algorithm enumerates a combinatorial space and gets reduced to satisfiable solutions.  Ogihara and Ray's algorithm constructs a space of potential solutions and eliminates non-satisfiable paths.  The Distribution algorithm constructs a set of non-conflicting states for a satisfiable solution.  Chapters 3 and 4 discuss the implementation of these algorithms.
				
\section{Simulation environment and physical devices}
	
%		<Paragraph> Introduce experiment
This project introduces an environment for simulating three molecular algorithms for solving {\sc Satisfiability}.  The environment provides standard operations for molecular computing that we introduce in Chapter 2. Chapter 7 provides results for a sweep of random {\sc Satisfiability} instances from the experimental setup of the simulation described in Chapters 5 and 6.
%		<Paragraph> Introduce implementation

The implementation of the simulation environment simulates synthetic molecular interactions.  Collection of runtime metrics during execution provide metrics of operator usage.  The runtime metrics get saved after algorithm execution, allowing for analysis of the molecular algorithm performance.
	
%		<Paragraph> Introduce physical machine
Chapter 7 provides production techniques for a general purpose molecular computation architecture We describe a fabrication overview for application of current gene sequencing technologies.  These technologies include fabrication of nanopores and micropumps \cite{ionTorrent, oxfordNanopore, Liao_Lee_Liu_Hsieh_Luo_2005}.

The next chapter begins with a background of molecular computation.  This introduces the terminology and techniques for molecular computation. Techniques from nanotechnology, microbiology and theoretical computer science get introduced for applied molecular computation.

\chapter{Background}

%<Paragraph> Overview of background

This chapter provides a background on molecular computation techniques.  We begin with an introduction to nanotechnology and then provide an example of encoding information with molecular matter.  Following this example, we introduce Adleman's molecular operators for solving an instance of {\sc Hamiltonian Path}.  The operators provide a base instruction set for molecular computing, and provide the primitives to construct molecular algorithms.

Finally, we provide an introduction to {\sc Satisfiability}.  We define {\sc Satisfiability} as a circuit.  We then view {\sc Satisfiability} as a language.  We also discuss practical matters related to efficiently evaluating {\sc Satisfiability}, such as how to encode input and output, and how to classify instances of {\sc Satisfiability} for the test cases that we consider.

\section{On nanotechnology and construction of molecules}

%		<Paragraph> Richard Feynman [introduces] nanotechnology
		
	Richard Feynman founded the field of nanotechnology in his 1959 talk `There's Plenty of Room at the Bottom' \cite{feynman1959}.  Examples of applied nanotechnology include the manufacturing of graphene \cite{Stankovich_Dikin_Dommett_Kohlhaas_Zimney_Stach_Piner_Nguyen_Ruoff_2006} and DNA nanopores \cite{dnaTransistorIBMpressrelease}. Graphene consists of an arrangement of carbon atoms that provides desirable physical and electrical properties \cite{Stankovich_Dikin_Dommett_Kohlhaas_Zimney_Stach_Piner_Nguyen_Ruoff_2006}. DNA nanopores create a physical channel for threading DNA for read and write operations.  Diverse applications continue to take advantage of the properties of nanotechnology.  Gene sequencing technologies provide an example of applied nanotechnology \cite{ionTorrent, oxfordNanopore}.  					
			%The work has driven the fields of molecular and quantum computation, VLSI circuit construction, and continues with innovative design and applications into many everyday processes.
		
%		<Paragraph> DNA substrate [builds] molecular definition
		
Smaller and cost-effective DNA sequencers provide the ability to read the contents of a gene.  Benchtop sequencers \cite{ionTorrent, oxfordNanopore} allows doctors to treat patients at the genome level from their office.  Life Technologies and Oxford Nanopore offer gene sequencers based on solid-state semiconductor technology \cite{ionTorrent, oxfordNanopore}.	

\section{On microbiology and computation}

%	<Paragraph> Microbiology [studies] molecular life
	
	Microbiology studies the interactions among organic molecules.  In this project, we explore techniques from applied genetics as a means for generalized computation.  Molecular computation encodes data as sequences of DNA or RNA.  
	
%	<Paragraph> Genetic alphabet [defines] life
		
%	The genetic alphabet defines a universal medium for proteins.  Each of the nucleotides (A, C, G, T, U) transcribes redundant encodings of amino acids.  Sequences of amino acids form structure as proteins.  Redundant encodings in the each amino acid permit syntax errors without a functional abnormality.  This redundant encoding structure permits mutation in the third nucleotide of each amino acid without consequence on the entire string.

%	<Paragraph> Complex molecules [contain] unique strings 
	
	Strings of nucleotides encode information as oligonucleotides.  A \textit{oligonucleotide} is a short string of genetic information.  There are several configurations for DNA and RNA; these include $+$RNA, $-$RNA, $+$DNA, $-$DNA, $\pm$RNA, $\pm$DNA, and +mRNA \cite{baltimore1971exp}.  The polarity of the DNA sequence denotes the direction of genetic information.  $+$DNA gets denoted by $5'$---$3'$ and $-$DNA gets denoted by $3'$---$5'$.  We focus on $+$DNA and $-$DNA as a substrate for encoding configurations for computational states.
	
%	<Paragraph> Each string [builds] structure from proteins
	Arbitrary encodings that represent mappings from variables to physical oligonucleotides may have undesirable structure and functionality.  Conventional techniques for DNA computing employ variable mappings from a library of oligonucleotides.
	%These strings avoid undesirable configurations, such as, self-complementary sequences that form hairpins \cite{dnaComputingModels2008}.
	
%	<Paragraph> Molecular definition [constructs] machine
Let us consider two techniques for representing information with oligonucleotides.  These allow us to encode integer mappings as a fixed width integer sequence.

Representing an integer sequence requires a systematic mapping of an oligonucleotide entry with an integer counterpart.  A fixed width representation map independent sequences on a readable boundary.  Now we explore an example for encoding an integer sequence with a sequence of oligonucleotides.  A sample mapping is provided in Table \ref{integer2OligoTable}.

\begin{table}[htdp]
\caption{A mapping of the integers $[0,5]$ with arbitrary oligonucleotide definitions.}
\begin{center}
\begin{tabular}{|c|c|c|}
\hline
 \textbf{Integer} & \textbf{Oligonucleotide} & \textbf{Reverse-complement}\\ \hline
0 & $5'$TCTCCC$3'$ & $3'$AGAGGG$5'$ \\
1 & $5'$AAACCC$3'$ & $3'$TTTGGG$5'$ \\
2 & $5'$GGTAAA$3'$ & $3'$CCATTT$5'$ \\
3 & $5'$CCCTCC$3'$ & $3'$GGGAGG$5'$ \\
4 & $5'$CTTTTC$3'$ & $3'$GAAAAG$5'$ \\
5 & $5'$CCTTCC$3'$ & $3'$GGAAGG$5'$ \\ \hline
\end{tabular}
\end{center}
\label{integer2OligoTable}
\end{table}%

Suppose that we would like to encode the sequence of integers $S$ as an equivalent oligonucleotide representation $O_1$.

We have, e.g., 
\[
S = [1, 3, 4, 3, 2, 0]
\]
and 
\[
O_1 = 5'\text{AAACCC}\mid \text{CCCTCC}\mid \text{CTTTTC}\mid \text{CCCTCC}\mid \text{GGTAAA}\mid \text{TCTCCC}3'.
\]
Recovering the sequence $S$ from $O_1$ can be done several ways.  Because the definition of the sequence exists, we may use the reverse complement to match sequences.  Another method splits the sequence $O_1$ on the encoding width.  In this case, the encoding width is six base pairs.  Gene sequencing tools permit reading the sequence and interpretation of the data with Table \ref{integer2OligoTable}.

%	<Paragraph> Interactions of molecules [performs] computation
Molecular computing encodes genetic information for both storage and operations on a problem state.  These interactions include matching and replication.  Although this setting describes and artificial construction for a machine, the natural encodings of organisms also share the mechanics that we exploit.  Interactions of molecules provide mechanics for generalized computation with oligonucleotides.
	
%	<Paragraph> Satisfiability [permits] universal computation
In the following chapters, we describe molecular algorithms for {\sc Satisfiability} and provide insight to construction of a generalized molecular computer.  Next we provide a toolbox for molecular computation.  The tools presented permit generalized computation with molecular biology techniques.  In the next section we introduce the techniques from Adleman's molecular toolbox \cite{Adleman:1994:MCS:189441.189442}.

\section{Adleman's molecular toolbox for solving {\sc Hamitonian Path}}
	
%	<Paragraph> Leonard Adleman [performs] first molecular computation 
Leonard Adleman performed the first molecular computation in 1994 with recombinant DNA in a bench laboratory setting \cite{Adleman:1994:MCS:189441.189442}.  This experiment solved a six vertex instance of {\sc Hamiltonian Path}, an \textsf{NP-complete} problem.  In this section, we describe the techniques used in this experiment. We provide definitions for the following operations from Adleman's molecular toolbox: append, extract, mix, split, and purify.
%	<Paragraph> Molecular computation [encodes] information from graph with DNA

\begin{definition}
{\sc Hamiltonian Path} \\
Given an undirected graph $G$, does there exist a path that visits every vertex exactly once?

\end{definition}

%


Adleman's encoding for graphs uses oligonucleotides for defining each vertex.  The vertex representation shares a similar definition for our example of encoding a sequence of integers in Table \ref{integer2OligoTable}.  Representing edges requires a definition of a reverse-complement oligonucleotide; this string connects the suffix of the vertex $v_i$ with the prefix of $v_j$.  For example, let us consider an example for appending $v_2$ to $v_1$.  Let, e.g.,
\begin{align*}
 v_1 &= 5'\text{ATCTTT}3' \\
 v_2 &= 5'\text{CCTATA}3'.
\end{align*}

\noindent From the definition of $v_1$ and $v_2$, we can construct an edge $e_{1,2}$ as
\[
e_{1,2} = 3'\text{AAAGGA}5'.
\]

\noindent Appending $v_2$ to $v_1$ gets accomplished by first attaching the edge $e_{1,2}$ to the vertex $v_1$
\begin{align*}
 5'\text{ATC}&\text{TTT}3' \\
  3'&\text{AAAGGA}5'.
\end{align*}

\noindent Next we attach $v_2$ to the resulting complex, yielding
\begin{align*}
 5'\text{ATCTTT}|&\text{CCTATA}3'\\
  3'\text{AAA}&\text{GGA}5'.
\end{align*}

\noindent Finally the edge may be removed and we have the sequence
\[
v_1 \cdot v_2 = 5'\text{ATCTTT}|\text{CCTATA}3'.
\]

The sequence $v_1 \cdot v_2$ represents the path $v_1$ to $v_2$, and can be obtained with the \textit{append} operation.  Possible solutions get stored in a test tube $T$.  $T$ begins as an empty tube.  For solving {\sc Hamitonian Path}, we introduce equimolar portions of each oligonucleotide vertex for a starting configuration with the \textit{mix} operation.

\begin{definition}
\textit{Mix}\\
$ T \leftarrow \text{mix}( T_1, T_2)$ --- combines two test tubes of information.  The output consists of a single set $T = T_1 \cup T_2$.
\end{definition}

A small initial set may be amplified with \textit{polymerase chain reaction} (PCR).  PCR thermocycles the contents of the tube to replicate the contents. Possible paths get randomly generated by introducing the vertex representation to the contents.  We create this representation elongating the initial vertex with a fixed path length.

\textit{Append} attaches a string to each string contained in a test tube.  \textit{Split} portions a tube into multiple portions.  We will use split-mix synthesis as a technique for generation of combinatorial space in Chapter 3.
%A possible path is created by randomly appending sequences to create a uniform statistical distribution of all paths.  

\begin{definition}
\textit{Append}\\
$T' \leftarrow \text{append}( T, a)$ --- the concatenation of the oligonucleotide $a$ with each element in $T$.  
\end{definition}

\begin{definition}
\textit{Split}\\
$[T', T''] \leftarrow \text{split}( T)$ --- distributes $T$ into two tubes.  Each of the resulting tubes, $T'$ and $T''$,  contain the same representative elements of $T$.
\end{definition}

From the tube $T$, we keep paths that begin with $V_{in}$ and end with $V_{out}$.  This ensures that the initial and terminal conditions for the graph get satisified.  Extracting only strings from $T$ that match these conditions reduces the number of potential strings.

\begin{definition}
\textit{Extract}\\
$ T' \leftarrow \text{extract}( T, a, x)$ --- separates all oligonucleotides from $T$ containing the sequence $a$ at position $x$.  The output consists of a set $T'$ of those oligonucleotides containing $a$ at position $x$.
\end{definition}

The tube $T$ consists of possible encodings that have the correct starting and ending vertices. We select only strings with length $n$, where $n$ is the number of vertices in $G$, to ensure that all vertices get traversed.  This can be performed with \textit{gel electrophoresis}, a technique for sorting molecules by mass.
Next, we ensure that each vertex occurs exactly once.  This gets accomplished by extracting possible vertices.  If a vertex occurs multiple times in a path, then the string representation gets discarded.

Finally, we check $T$ with \textit{detect} to determine if any valid paths remain.  If valid paths exist, then each string may be read for the path assignment.

\begin{definition}
\textit{Detect}\\
$ \text{detect}( T)$ --- determine if any encodings are present in $T$.  The output consists of $true$ or $false$, for $T \neq \emptyset$ or $T = \emptyset$ respectively.
\end{definition}

\subsection{Additional molecular operators}

In the following chapters, we will use the molecular operators for construction of molecular {\sc Satisfiability} solvers.  The Distribution algorithm, introduced in Chapter 4, requires the \textit{splice} operation.
\begin{definition}
\textit{Splice}\\
$[a_1, a_2] \leftarrow \text{splice}(a, b)$ --- cuts an oligonucleotide $a$ with a subsequence $b$ into two pieces by a restriction enzyme.  These two pieces are $a_1$ and $a_2$.
\end{definition}

In the implementation of a simulation system, we avoid redundant string representations with the \textit{purify} operation.  This is a synthetic version of PCR.  Purify balances the space representation of molecules with a uniform distribution.
\begin{definition}
\textit{Purify}\\
$T' \leftarrow \text{purify}(T)$ --- provides a uniform distribution from the contents of $T$ as $T'$.
\end{definition}

%	<Paragraph> Execution [solves] NP-complete problem instance	
%	<Paragraph> Encoding [requires] permutations of space	
%	<Paragraph> Subsequent molecular algorithms [constrain] required space

%\begin{definition}
%$[a_1, a_2] \leftarrow \text{splice}(a, b)$ --- is defined as cutting a string $a$ with a subsequence $b$ into two pieces by a restriction enzyme.  These two pieces are $a_1$ and $a_2$.
%\end{definition}
	
\section{Definition of {\sc Satisfiability}}

%	<Paragraph> Introduce and motivate {\sc Satisfiability}
	
{\sc Satisfiability} is a canonical \textsf{NP-complete} language.  Each {\sc Satisfiability} instance efficiently encodes a set of conditions to satisfy.  Because {\sc Satisfiability} is \textsf{NP-complete}, it can be reduced to any \textsf{NP-complete} language.

\begin{definition}
{\sc Satisfiability}\\
Formally defined as the language
\[
\text{\sc Satisfiability} = \{ \langle \phi \rangle \mid \phi \text{ is a satisfiable Boolean formula}\} \cite{sipser06}.
\]	
\end{definition}

%	<Paragraph>	Define {\sc Satisfiability} with circuit 

Evaluation of a {\sc Satisfiability} instance requires validating the input with the instance definition.  We introduce {\sc Satisfiability} evaluation with a circuit.  Let us consider a three-layered circuit for {\sc Satisfiability}.  This circuit consists of $n$ inverters, $m$ \textbf{OR} gates, and one \textbf{AND} gate with $m$-fan-in.  This circuit behaves according to the internal wiring of the input expression $\phi$. Figure \ref{blackBoxSat} contains a schematic for {\sc Satisfiability}.	
%	<Figure>	Circuit description
\begin{figure}[htbp]
\begin{center}

	\includegraphics[width=0.9\textwidth]{figures/circuitLabeled.jpg}

\caption{A circuit describing {\sc Satisfiability}.}
\label{blackBoxSat}
\end{center}
\end{figure}
	
\FloatBarrier

%	<Paragraph> Provide complexity of {\sc Satisfiability}
The realization of {\sc Satisfiability} as a circuit shows two insights of the problem.  {\sc Satisfiability} can be implemented with logic components proportional to the problem size, and the worst case verification consists of enumerating all possible switch configurations.  {\sc Satisfiability} as a language demonstrates that it is equivalent to all other \textsf{NP-complete} languages.

Cook and Levin independently introduced the canonical instance of a \textsf{NP-complete} language {\sc Satisfiability}\cite{Cook:1971:CTP:800157.805047, levin1973}.  A \textsf{NP-complete} language is one that is in \textsf{NP} and \textsf{NP-hard}.  A \textsf{NP-hard} language is at least as hard as any problem in \textsf{NP}.

%	<Paragraph> Motivate practical input and classifying metrics 

The next section considers standards adopted for {\sc Satisfiability}.  This allows practitioners to apply {\sc Satisfiability} in various settings.
	
\section{Evaluating {\sc Satisfiability}}

%	<Paragraph> Overview of evaluation
	
	In this section, we describe two standards for encoding the {\sc Satisfiability} problem that we adopt for the implementation.  This includes the input and output standards for the {\sc Satisfiability} Competition \cite{dimacsFormat, satcompetition}.  
	
	Next, we introduce problem instance classification for {\sc Satisfiability}.  Classification of {\sc Satisfiability} problem instances include randomly generated, combinatorial, and industrial \cite{satcompetition}.  The experimental setup in Chapter 6 considers generation of random $k$-{\sc Sat} input.

	\subsection{Input and output}
	
%		<Paragraph> {\sc Satisfiability} standards [provide] common interface
	
%  Conforming to standards allows datasets and common interfaces to be shared.  
 Each year a competition showcases techniques for evaluating {\sc Satisfiability} \cite{satcompetition}.  We conform to the standards of the {\sc Sat} Competition \url{http://www.satcompetition.org/}.  {\sc Sat} solvers demonstrate state-of-the-art techniques for solving three main tracks of {\sc Satisfiability} instances.  The tracks exhibit applications for {\sc Satisfiability}, including: industrial applications, hard combinatorial, and random problem instances.
 
 The input and output standards for {\sc Satisfiability} allow common benchmarks for {\sc Sat} solvers.
	
		\subsubsection{Input}
		
%			<Paragraph> DIMACS CNF [provides] standard benchmark instances
DIMACS CNF provides a standard input for {\sc Satisfiability} \cite{dimacsFormat}.  The format permits sharing of existing {\sc Satisfiability} benchmarks by encoding {\sc Satisfiability} in conjunctive normal form (CNF).  The format is user readable with a natural encoding for {\sc Satisfiability}.  We provide an example of this encoding in Section \ref{inputSection}.
		
		\subsubsection{Output}
		
%			<Paragraph> Sat Competition output [provides] standard output

{\sc Sat} Competition output consists of the status for a DIMACS CNF input instance \cite{satcompetition}.  This includes the known state, either \texttt{SATISFIABLE}, \texttt{UNSATISFIABLE}, or \texttt{UNKNOWN}.  When a witnessing satisfying assignment occurs, the assignment gets provided as a list of integers with the \texttt{SATISFIABLE} state.  We provide an example along with a custom interface in Section \ref{outputSection}.
	
	\subsection{Metrics for classifying {\sc Satisfiability}}

%		<Paragraph> Describe metrics
{\sc Sat} phase transition and {\sc Sat} backbones are two classifying metrics for {\sc Satisfiability}.  These metrics may be used to classify {\sc Satisfiability} expressions.  We will use these metrics in the next section for defining a collection of random $k$-{\sc Sat} instances.
		
%		<Paragraph> {\sc Sat} phase transition

\begin{definition}
CNF\\
Conjunctive Normal Form consists of the intersection of sets of disjunctive literals. 
\end{definition}

\begin{definition}
$k$-CNF\\
Consists of a CNF expression with each disjunctive clause containing $k$ literals.
\end{definition}

\begin{definition}
$k$-{\sc Sat}\\
Problem variant of {\sc Satisfiability} where each clause consists of $k$ Boolean literals.  $k$-CNF formula provide an equivalent representation.
\end{definition}

\begin{definition}
{\sc Sat} phase transition\\
The {\sc Sat} phase transition is a region where both satisfiable and unsatisfiable instances are likely.  The ratio of clauses to variables $\alpha = m/n$ provides a characterization for where phase transitions may occur in the space of all $k$-CNF formula \cite{Doherty08thehandbook,Gent94thesat}.
\end{definition}
		
%		<Paragraph> {\sc Sat} backbones
\begin{definition}
{\sc Sat} backbones\\
{\sc Sat} backbones are the variable assignments present in all of the satisfying assignments to a {\sc Satisfiability} expression \cite{Zhang2001}.  This is a set of variables that occur in all satisfiable valuations for an input expression.  

\end{definition}



	\subsection{{\sc Satisfiability} instances}
		
%		<Paragraph> Various methods for [constructing] {\sc Satisfiability} instances
There are several methods for constructing {\sc Satisfiability} instances.  We consider techniques for constructing instances based on random assignment, combinatorial, and real applications from industry.  The instance type demonstrate properties of {\sc Satisfiability} and provide heuristics for certain input.
	
%		<Paragraph> {\sc Satisfiability} instance [generated] from random assignment
A random $k$-{\sc Sat} expression consists of $m$ clauses with $k$ literals per-clause from  $n$ variables \cite{wilsonKsat}. Variable assignments get distributed with probability $Prob\left(\frac{1}{n}\right)$.  The positive or negative variable polarity get assigned with a probability $Prob\left(\frac{1}{2}\right)$.

%Random $k$-{\sc Sat} are generated with \texttt{ksat.c} \cite{wilsonKsat}.  A hash of the clause representation ensures that all clauses are independent \cite{wilsonKsat}.

%During generation of these formulas a hash is implemented to ensure that the random assignments ensure independent clauses and non-redundant variable assignments.  
		
%		<Paragraph> {\sc Satisfiability} instance [constructed] as hard assignment

Combinatorial instances provide difficult benchmark cases.  These instances can be converted from other \textsf{NP-complete} problems.  This category also includes games and graph theoretic problems represented as {\sc Satisfiability}. 
		
		
%		<Paragraph> {\sc Satisfiability} instance [applied] from real world problems

Industrial processes apply {\sc Satisfiability} in many real world problems.  This includes circuit layout, planning, logistics, circuit fault testing and many other industrial \textsf{NP-complete} problems.  Applications for industrial {\sc Sat} will often apply heuristics, and approximation techniques to relax the problem.  This allows approximate solutions to be computed in an efficient amount of time.
		
\chapter{Existing molecular algorithms for {\sc Satisfiability}}

%<Paragraph> Introduce two molecular algorithms for {\sc Satisfiability}

In this chapter, we introduce two molecular algorithms for {\sc Satisfiability}.  These algorithms are distinct in the resolution of a {\sc Satisfiability} instance.  Lipton's algorithm requires a space to be constructed before execution, where Ogihara and Ray's algorithm constructs a valid space during execution.  Following the description, we explore the physical implementation and simulation of these algorithms.
\section{Lipton's algorithm for {\sc Satisfiability}}

%	<Paragraph> Introduce Lipton's algorithm
Introduced in 1995 by Richard Lipton \cite{Lipton95usingdna}, this algorithm creates an exponential search space for the CNF expression.  Each variable gets evaluated with the combinatorial space, reducing the space on each iteration.  The satisfiable configurations are present in the remaining space.  This algorithm is analogous to a conventional brute-force search for all solutions. 	
	\subsection{Description of Lipton's algorithm}
		
%		<Paragraph> Describe preconditions
Lipton's algorithm consists of two main procedures.  The first phase constructs a combinatorial space of $2^n$ independent vectors.  Second, the combinatorial space gets reduced based on the input CNF instance. 		

%		<Paragraph> Describe setup
The function {\sc Combinatorial Generate}$(n)$ implements the split-mix synthesis technique \cite{furka1982, furkaBook}.  It returns a gel consisting of $2^n$ independent oligos that correspond to a unique vector space.  The space begins construction with an initial medium.  An iterative loop elongates a growing solution with the split-mix synthesis.  Each split corresponds with appending the tubes with a truth and false assignment.  The two tubes are mixed and amplified to contain equimolar portions.  

%		<Paragraph> Describe execution
The amplification process gets modeled with a purification step.  This eliminates all redundant strings for the simulated environment.  After the iteration completes, the complete combinatorial space gets returned. This space consists of $2^n$ vectors of length $n$. 

%		<Paragraph> Describe Output
From the combinatorial space, we will begin to filter satisfying solutions to the input CNF formula.  For each clause, we extract each of the variables present in the solution space.  A disjunctive set $T_C$ contains the satisfied string instances for each clause.  {\sc Lipton's Algorithm} iterates over each of the clauses.  From the selected clause, the variables get extracted from the combinatorial space.  Once complete, the remaining space, $T$, contains satisfiable instances for $\phi$.
%\par Introduced in 1995 by Lipton \cite{Lipton95usingdna}, this algorithm creates an exponential search space for the CNF expression.  Once the space is created, each variable is evaluated and the space is reduced to only the solutions that satisfy all remaining strings.  This algorithm is analogous to a conventional brute-force search for all solutions.
%
%
%%%%%
%\subsection{Description}
%This algorithm consists of two main procedures.  Upon execution of the algorithm a combinatorial space of $2^n$ independent vectors encoding possible solutions.
%
%Construction of this space employs a technique from combinatorial chemistry, split-mix synthesis.  Split-mix synthesis was invented in 1982 by \'{A}rp\'{a}d Furka, and later adopted to assist in the generation of synthetic molecules \cite{furka1982}.
%
%The function {\sc Combinatorial Generate}$(n)$ implements the split-mix synthesis technique.  It returns a gel consisting of $2^n$ independent oligos that correspond to a unique vector space.  The space begins construction with an initial medium.  Upon each iteration the space is amplified and split into two equal portions.  Each set of sequences is then extended by a value representing a truth and a false assignment to a variable.  
%
%Upon each iteration the gel is purified, that is eliminate all redundant strings.  Once each variable extends the growing vector space, a complete representation of the space is returned.  This space consists of $2^n$ vectors of length $n$.
%
%From the combinatorial space, we will begin to filter satisfying solutions to the input CNF formula.  For each clause, we extract each of the variables present in the solution space.  A disjunctive set $T_C$ contains the satisfied string instances for each clause.  This process continues until each of the clauses are read and its variables are extracted.  The remaining space, $T$, contains satisfiable instances for $\phi$.
%
	
	\subsection{Pseudocode for Lipton's algorithm}
	
%	<Paragraph> Introduce pseudocode
Algorithms 3.1.1 and 3.1.2 provide pseudocode for Lipton's algorithm.  
	
	
%%%%%%%%%%%%%%%%%%%%%%%
%%%%%%%%%%%%%%%%%%%%%%%
\begin{pseudocode}{Lipton's Algorithm}{\phi}
n \text{ number of variables in } \phi \\
\\
T \GETS \text{{\sc Combinatorial Generate}}(n) \\
\FOREACH \text{clause } C \text{ in } \phi \DO
	\BEGIN
	T_c \GETS \emptyset \\
	\FOREACH \text{variable } v \text{ in } C \DO
		\BEGIN
			\IF v \text{ is a positive literal} \THEN
				\BEGIN
					T_P \GETS \text{extract}(T, +v)\\
					T_c \GETS \text{mix}(T_P, T_c)						
				\END
			\ELSE
				\BEGIN
					T_N \GETS \text{extract}(T, -v)\\
					T_c \GETS \text{mix}(T_N, T_c)						
				\END
		\END
	\\
	T \GETS T_c\\
	\END
\\
\RETURN{\text{detect}(T)}
\end{pseudocode}


	
\section{Ogihara and Ray's algorithm for {\sc Satisfiability}}

%	<Paragraph> Introduce Ogihara and Ray's algorithm

Ogihara and Ray's algorithm consist of a breadth-first evaluation of clauses from a CNF formula \cite{Ogihara:1996:BFS:898228,Ogihara97dna-basedparallel}.  The algorithm constructs a set of potential solutions based on parsing a 3-CNF formula.  In this section, we describe the preconditions and execution of Ogihara and Ray's algorithm.

\subsection{Description of Ogihara and Ray's algorithm}
		
%		<Paragraph> Describe preconditions
Prior to execution of the algorithm it requires two attributes of CNF input:

\begin{enumerate}
\item All clauses consist of exactly three literals
\item All clauses must be sorted by variable
\end{enumerate}

Considering only $3$-{\sc Sat} expressions ensure attribute (1) gets fufilled.  If models of $k$-{\sc Sat} with $k > 3$, then a reduction to $3$-{\sc Sat} must occur prior to execution.

Prior to execution of the algorithm, the parsing of the DIMACS CNF input gets sorted.  This step ensures that attribute (2) gets satisfied.  Providing the weak ordering

\[
v_1 < \cdots < v_n,
\]

where the polarity of each variable may consist of a positive or negative valuation.

%		<Paragraph> Describe setup

The initial tube consists of potential states for the first two variables.

%		<Paragraph> Describe execution

Expanding each partial assignment iterates over each clause in the input CNF.  Construction of satisfiable expressions consider the possibilities of the clause ordering

\[
x_u < x_v < x_w.
\]

{\sc Ogihara and Ray's Algorithm} evaluates each subsequent variable and determines possible assignments.  The possible assignments for the variables $v_1$ and $v_2$ get extracted if $v_3$ matches.  Effectively pruning only potential solutions.  These potential solutions $T_P$ and $T_N$ get appended with the positive ($POS$) or negative ($NEG$) string assignments.  The algorithm continues until each variable gets evaluated.  

The remaining space $T$, once the algorithm terminates, contains all solutions for the CNF instance $\phi$.



%		<Paragraph> Describe Output


%\par Originally introduced in 1996 by Ogihara \cite{Ogihara:1996:BFS:898228}, and extended in 1997 with Ray \cite{Ogihara97dna-basedparallel}, this algorithm builds a solution space with a breadth-first search evaluation.  Prior to execution to the algorithm, each of the clauses variable's in the CNF expression are sorted.  Experimental results of a physical algorithm implementation for this algorithm was conducted by Yoshida et al. [23].
%
%
%%%%%
%\subsection{Description}
%Ogihara and Ray's algorithm consists of a breadth-first evaluation of clauses from a CNF formula.  Prior to execution of the algorithm it requires two attributes of CNF input:
%
%\begin{enumerate}
%\item All clauses consist of exactly three literals
%\item All clauses must be sorted by variable
%\end{enumerate}
%
%Attribute (1) is ensured by considering only $3$-{\sc Sat} expressions.  If models of $k$-{\sc Sat} with $k > 3$, then a reduction to $3$-{\sc Sat} must occur prior to execution.
%
%Attribute (2) is ensured prior to execution of the algorithm.  This is done during the parsing of the DIMACS CNF input.  Further, we only provide valid $3$-CNF input.  Sorting ensures that the weak ordering is maintained
%\[
%v_1 < \cdots < v_n,
%\]
%
%where the polarity of each variable may consist of a positive or negative valuation.
%
%An initial tube is created with a base representation for the first two variables.
%
%Expanding each partial assignment iterates over each clause in the input CNF.  Construction of satisfiable expressions consider the possibilities of the clause ordering
%\[
%x_u < x_v < x_w.
%\]
%
		
	\subsection{Pseudocode for Ogihara and Ray's algorithm}

%	<Paragraph> Introduce pseudocode
	
Algorithm 3.2.1 provides pseudocode for Ogihara and Ray's algorithm.
	
	
	
\begin{figure}[htbp]
	\renewcommand{\figurename}{Algorithm}
	\renewcommand{\thepseudocode}{\ref{ogiharaRayAlgorithm}}
	
	\begin{center}

	\begin{pseudocode}[shadowbox]{Ogihara and Ray's Algorithm}{\phi}
	
	\text{// Input $\phi$ consists of $n$ variables.}\\
	\text{// Each clause $C$ contains ordered literals $(a,b,c)$.}\\
	\text{// Extract literals using the condensed notation $v_P$, where: }\\
	\text{// \hspace{1em} $v_P$ matches the literal, and $v_N$ matches the negated literal.}\\
	\\
	T \GETS \{ \texttt{STT}, \texttt{STF}, \texttt{SFT},  \texttt{SFF}\} \\
	
	\FOR v \GETS 3 \text{ to } n \DO
		\BEGIN
		[T_P, T_N] \GETS \text{split}(T)\\
	
		\FOREACH \text{clause } C \text{ in } \phi \DO
			\BEGIN
				(a, b, c) \GETS C\\
				\IF v_{\texttt{T}} = c  \THEN
					\BEGIN
						T_{P1} \GETS \text{extract}(T_N, a_P)\\
						T_{N1} \GETS \text{extract}(T_N, a_N)\\				
						T_{P2} \GETS \text{extract}(T_{N1}, b_P)\\
						T_{N} \GETS \text{mix}(T_{P1}, T_{P2})\\
						T_{N} \GETS \text{purify(}T_{N}\text{)}					
					\END \\  
				\IF v_{\texttt{F}} = c \THEN
					\BEGIN
						T_{P1} \GETS \text{extract}(T_P, a_P)\\
						T_{N1} \GETS \text{extract}(T_P, a_N)\\				
						T_{P2} \GETS \text{extract}(T_{N1}, b_P)\\
						T_{P} \GETS \text{mix}(T_{P1}, T_{P2})\\
						T_{P} \GETS \text{purify(}T_{P}\text{)} 						
					\END\\
			\END\\
			T_P \GETS \text{append}(T_P, v_{\texttt{T}})\\
			T_N \GETS \text{append}(T_N, v_{\texttt{F}})\\
			T \GETS \text{mix}(T_P, T_N)\\
			T \GETS \text{purify(}T\text{)} \\									
		\END\\
	\RETURN{\text{detect}(T)}
	\end{pseudocode}

\caption{{\sc Ogihara and Ray's Algorithm} evaluates each subsequent variable and determines possible assignments.  The possible assignments for the variables $a$ and $b$ get extracted if $c$ matches the current variable $v$.  Effectively pruning only potential solutions.  These potential solutions $T_P$ and $T_N$ get appended with the positive or negative string assignments.  The algorithm continues until each variable gets evaluated.  The remaining space $T$ contains all solutions for the CNF instance $\phi$ after the algorithm terminates.}
\label{ogiharaRayAlgorithm}
\end{center}
\end{figure}


	

\section{Implementations of molecular {\sc Satisfiability} solvers}

	In this section, we describe physical and simulated implementations for molecular {\sc Satisfiability} algorithms.  This includes simulation of Lipton's and Ogihara and Ray's algorithms.  We see a physical implementation of Ogihara and Ray's algorithm with manual laboratory procedures.

	\subsection{Physical implementations}
	
%	<Paragraph> Describe laboratory 

Yoshida and Suyama implemented Ogihara and Ray's algorithm with manual molecular biology techniques \cite{dnaBasedImplemetation_Yoshida2000}.  This experiment solved a 3-CNF instance with four variables and 10 clauses.

	
	\subsection{Simulations}

%		<Paragraph> Describe computer simulation 
Martín-Mateos et al. introduced a simulation for Lipton's algorithm \cite{MartinMateos02molecularcomputation}.   Molecular operations get implemented with ACL2, a Common Lisp variant.  The framework for this environment implemented test cases for Lipton's algorithm.

Ogihara provides test results for implementation of his original molecular algorithm \cite{Ogihara:1996:BFS:898228}.  This simulation provides a comparison with Lipton's algorithm for practical length restrictions.


\chapter{A new molecular algorithm for {\sc Satisfiability}}

%<Paragraph> Introduce 
This chapter introduces a new molecular algorithm for {\sc Satisfiability}.  The distribution algorithm parses an input CNF expression into growing and self regulated set of possible combinations.
\section{Distribution algorithm for {\sc Satisfiability}}

%	<Paragraph> Introduce Distribution algorithm	
The distribution algorithm parses an input CNF expression into growing and self regulated set of possible combinations.  A possible combination begins with all members of the first clause.  Variables get inserted into an expanding set of valid assignments.  A clause gets eliminated when an assignment contains a conflict.
	\subsection{Description of the Distribution algorithm}
		
%		<Paragraph> Describe preconditions	
%		<Paragraph> Describe setup
		
Initially the algorithm starts with the variable assignments of a clause.  Evaluation of subsequent clauses extends the solution space with the {\sc Insert Variable} subroutine.  During each insertion, the variable gets inserted into a potential solution vector.  Table \ref{distributionInsertTable} lists the four possibilities for variable assignment.

\begin{table}[htdp]
\caption{Configurations for the {\sc Insert Variable} subroutine}
\begin{center}
\begin{tabular}{|c|c|c|}
\hline
Item & Return state & State \\ \hline 
1	& $v \cdot s$ & if $v$ is less than all elements in $s$ \\ 
& &  \\ \hline
2	& $s \cdot v$ & if $v$ is greater than all elements in $s$ \\ 
& &  \\ \hline
3	& $s_1 \cdot v \cdot s_2$ & if $v$ is between two elements in $s$ \\ 
& &  \\ \hline
4	& $\emptyset$ & if $v$ conflicts with $-v$ in $s$\\ \hline
\end{tabular}
\end{center}
\label{distributionInsertTable}
\end{table}%

\FloatBarrier

%		<Paragraph> Describe execution
		
During this phase, each variable from a disjunctive clause gets considered, incrementally constructing a partial solution space.  Items (1), (2) and (3) place a variable $v$ into an existing sequence $s$.  Each of these cases represents when the variable $v$ get inserted in a non-decreasing sequence.

A variable conflict occurs when both positive and negative assignments of a variable occur in a sequence $s$.  In this case, the sequence $s$ gets removed from the set potential solutions.

Redundant vectors get removed after insertion of the next disjunctive clause.  Any remaining valuations in the solution space contain non-conflicting variable assignments.  This does not immediately require that each valuation be a complete satisfiable assignment. Satisfiable valuations remain in a non-empty satisfying solution space.
		
		

%		<Paragraph> Describe Output



Vectors that are of equal magnitude of the number of variables in the problem instance are satisfiable valuations.  However, there may exist solutions that span only the required satisfiable assignments; that is activate each of the independent clauses with at least one non-conflicting assignment.  This assignment may be the minimum valuation for the expression, in the case that the backbone consists of the variables of the maximum valuation.

		
	\subsection{Pseudocode for Distribution algorithm}

Algorithms 4.1.1 and 4.1.2 provide pseudocode for the Distribution algorithm.  



\begin{figure}[htbp]
\begin{center}

	\begin{pseudocode}{Distribution Algorithm}{\phi}
	
	\FOREACH \text{clause } C \text{ in } \phi \DO
		\BEGIN 
				T_C \GETS \emptyset \\
			\FOREACH \text{literal } \ell \text{ in } C \DO
				\BEGIN
					T_I \GETS \text{\sc Insert Literal}(T, \ell)\\
					T_C \GETS \text{mix}(T_C, T_I)\\
				\END\\
				T \GETS \text{purify(}T_C\text{)} \\
		\END
	\\
	\RETURN{detect(T)}
	\end{pseudocode}

\caption{{\sc Distribution Algorithm} constructs a set of non-conflicting assignments for a CNF instance.  This algorithm inserts contents of each clause $C_i$ into $T$ with the {\sc Insert Literal} subroutine.}
\label{distributionAlgorithm}
\end{center}
\end{figure}




%
%\section{Simulation of distribution algorithm}
%
%\section{Physical construction of distribution algorithm}
%	





\chapter{Molecular Simulation: A system for molecular computation}

%	<Paragraph> Decribe overview of chapter contents

This chapter introduces Molecular Simulation: A system for molecular computation.  We provide an overview of the software and download location for Molecular Simulation and its documentation. We provide tools for use with Molecular Simulation.  This includes automated documentation, Perl execution scripts, and visualization for output data.  We provide examples for Molecular Simulation's input and output.  Invocation of Molecular Simulation from the command line provides user configurable options.  The next chapter describes the usage of Molecular Simulation with automated execution.
	
	\section{Overview}
	
%		<Paragraph> Define scope of simulation system

Molecular Simulation provides a molecular lab for operating on DNA.  The present simulation implements three molecular algorithms for {\sc Satisfiability}.  The included \texttt{Perl} scripts process DIMACS CNF input directories with invocations to Molecular Simulation.

Molecular Simulation may be executed directly or invoked with the assistance of an execution script.  The system requirements to execute or design a molecular experiment are listed in this section.  
		
This program is a simulated molecular lab for experimenting with DNA operations. Implementation of three molecular algorithms for solving {\sc Satisfiability} include Lipton's algorithm, Ogihara and Ray's algorithm, and the Distribution algorithm.  Chapters 3 and 4 provide a background and pseudocode for these algorithms.		
%		<Paragraph> Reference download and documentation locations

Molecular Simulation can be downloaded from \url{http://www.cs.rit.edu/~dnc6813/project/project.html}.  The archive contains example DIMACS CNF testbench and example instances.  This document and the online documentation may be used independently for getting started.

 	\section{Requirements}
 	
 	Requirements for Molecular Simulation are specified in this section.  This includes the hardware and software requirements for running Molecular Simulation on your system.

	\subsection{Hardware requirements}
	
	\par \noindent Molecular Simulation requires a 64-bit processor with 2 GB of RAM.  
 	
 	\subsection{Software requirements}
	
	\par \noindent \texttt{gcc} (GNU Compiler Collection) must be installed on your system. \\
	
	\par \noindent \texttt{Perl} must be installed on your system to automate build and execution of Molecular Simulation.
	

	\section{Documentation}

Molecular Simulation allows the user to execute DIMACS CNF input.  Usage for algorithm implementation permits the user to design and test new molecular algorithms.

Detailed documentation can be accessed from the project website.  This includes an overview of Molecular Simulation along with detailed class and function documentation.  Any modifications to this software should use this as a starting point in development. 
		
	\section{Tools}
	
%		<Paragraph> Describe external tools
This project uses several tools for automating tasks and execution.  In this section, we describe the tools for documentation, automated execution, and output visualization for Molecular Simulation.
		
		\subsection{Doxygen}
		
%			<Paragraph> Describe Doxygen formatted documentation

\texttt{Doxygen} can be used to generate automated documentation for Molecular Simulation.  The online and offline documentation get generated from Doxygen formatted documentation.  Download and learn \texttt{Doxygen} at \url{http://www.stack.nl/~dimitri/doxygen/}.			
		\subsection{Perl utilities}
		
%			<Paragraph> Describe Perl utilities				
			
The source directory includes several \texttt{Perl} scripts to assist in building and initiation of tests for Molecular Simulation.  Table \ref{perlScriptTable} documents the basic usage for build and testbench execution scripts. Each script provides detailed execution options.

\begin{center}
\begin{table}[htdp]
\caption{\texttt{Perl} execution commands and descriptions.}
\begin{center}
\begin{tabular}{| l | l | p{5.7cm} |}
\hline

\textbf{Perl script} & \textbf{Usage} & \textbf{Description} \\ \hline 
\texttt{build.pl} & \texttt{\$ perl build.pl} & Compiles Molecular Simulation and generates an executable in the directory \texttt{./execute/simulation}.\\ 
& & \\
\texttt{buildGenerate.pl} &\texttt{\$ perl buildGenerate.pl} &  Generates a sweep of CNF formulas over a range of $k$-{\sc Sat} ratios.  Program uses a modified random $k$-{\sc Sat} generator from Microsoft Research.\\ 
& & \\
\texttt{executeMolecularSat.pl} &\texttt{\$ perl executeMolecularSat.pl}  & Executes Molecular Simulation for a directory of {\sc Satisfiability} expressions with desired algorithms.  If no options are specified, then each of the three algorithms are executed and output is generated in the same test directory. \\ 
& & \\
\texttt{runSimulation.pl} & \texttt{\$ perl runSimulation.pl} & Executes \texttt{build.pl} followed by \texttt{executeMolecularSat.pl}.  Any command line arguments get passed to \texttt{executeMolecularSat.pl}\\ \hline

\end{tabular}
\end{center}
\label{perlScriptTable}
\end{table}%
\end{center}
			
			
			
		\subsection{Data Visualization}
		
%			<Paragraph> Describe Modified visualization from Ben Fry

We adopt a modified plot for visualization of output data.  Ben Fry's example in Chapter 4 of Visualizing Data\cite{fryVisualizingData} provides a framework for importing output from Molecular Simulation.  We provide the modified Processing source in the Molecular Simulation archive.  The visualization directory contains a README for usage.

%			<Paragraph> Discuss implementation
%In the following sections, we provide the output format for Molecular Simulation.  We explore several use cases, including a native desktop and online Javascript applications.
			
	\section{Input}
	\label{inputSection}
	
%		<Paragraph> Describe DIMACS CNF 

Input to Molecular Simulation consists of a DIMACS CNF file. The definition of the \texttt{*.cnf} filetype can be accessed from \url{ftp://dimacs.rutgers.edu/pub/challenge/satisfiability/doc/}.		
%		<Verbatim> Show input example

\begin{verbatim}
c comments begin with a `c'
c
c cnf input is designated with `p cnf'
c    followed by number of variables <n>, and clauses <m>
c
p cnf <n> <m>
c
c A clause is represented by a sequence of <k> integers,
c     separated by whitespace and ending with a `0'.
c Each variable is represented by the integer sequence, 
c    negative polarity is represented by `-'.
c
-3 9 14 0
6 -9 -12 0
-2 11 17 0
3 -13 -17 0
\end{verbatim}
		
	\section{Output}
	\label{outputSection}
	
%		<Paragraph> Describe Sat Competition output

Output from Molecular Simulation, by default, conforms to the 2011 Sat Competition rules.  The rules can be accessed from \url{http://www.satcompetition.org/2011/rules.pdf}.


%		<Verbatim> Show output example
\begin{verbatim}
c comments begin with a `c'
c
s SATISFIABLE
c
c A line beginning with a `s' marks the status.
c This can be either `UNSATISFIABLE', `SATISFIABLE', or `UNKNOWN'.
c
v -3 -9 11 13 0
c
c A satisfiable witness begins with a `v' and ends with a `0'.
c     Between the `v' and `0' is a sequence of integers encoding a satisfiable assignment.
\end{verbatim}

%		<Paragraph> Describe Output Options
Table \ref{outputTableDefiniton} describes an extended custom output.  This output reports parameters for metric performance evaluation.
\begin{table}[htdp]
\caption{Molecular Simulation output logging.}
\begin{center}
\begin{tabular}{| l | l |}
\hline
\textbf{Parameter} & \textbf{Description} \\ \hline	
\texttt{c algorithmType:}&	Display the algorithm type: \texttt{Lipton}, \texttt{Ogihara-Ray}, \texttt{Distribution}\\ 
\texttt{c algorithmTime:}&	Display the algorithm execution time in seconds.\\ 
\texttt{c solutionMemory:}& Display the solution space memory footprint in Bytes.	\\ 
\texttt{c mixCount:}	&	Display the number of \texttt{mixes} required during algorithm execution.\\ 
\texttt{c extractCount:}&	Display the number of \texttt{extracts} required during algorithm execution.\\ 
\texttt{c appendCount:}&	Display the number of \texttt{appends} required during algorithm execution.\\ 
\texttt{c splitCount:}	&	Display the number of \texttt{splits} required during algorithm execution.\\ 
\texttt{c spliceCount:}&	Display the number of \texttt{splices} required during algorithm execution.\\ 
\texttt{c purifyCount:}&	Display the number of \texttt{purifications} required during algorithm execution.\\ 
\texttt{c numVar:}	&	Display the number of \texttt{variables} in the input CNF expression.\\ 
\texttt{c numClause:}	&	Display the number of \texttt{clauses} in the input CNF expression.\\ \hline

\end{tabular}
\end{center}
\label{outputTableDefiniton}
\end{table}%
		

\FloatBarrier
			
\section{Execution}
	
%		<Paragraph>	Describe invocation of Molecular Simulation
Invocation of Molecular Simulation can be performed from the command line.

\[
\texttt{\$ ./execute/simulation i [input] [options]}
\]

The \texttt{[input]} consists of a DIMACS CNF file.  Command line \texttt{[options]} may be a combination of the options in Table \ref{MolecularCommandLineArgs}.

\begin{table}[htdp]
\caption{Command line options for Molecular Simulation}
\begin{center}
\begin{tabular}{|c|c|l|}
\hline
\textbf{Argument} & \textbf{Parameters} & \textbf{Description} \\ \hline
%\texttt{i}		& \texttt{[input]} & Required DIMACS CNF input \\
% 				&				   &		 \\
 \texttt{-a}	& 				   & Algorithm select \\
  				&				   &		 \\
 				& \texttt{d}	   & Distribution algorithm		 \\
 				& \texttt{l}	   & Lipton's algorithm		 \\
 				& \texttt{o}	   & Ogihara and Ray's algorithm		 \\
 				&				   &		 \\ \hline 				
\texttt{-d}		&				   & Debug		 \\ 				
 				&				   &		 \\ \hline
\texttt{i}		&				   & Input		 \\ 				
				& \texttt{[input]} & DIMACS CNF file		 \\ 				
 				&				   &		 \\ \hline 				
\texttt{-w}		&				   & Write output to file		 \\
 				& \texttt{[output]} & Output filename \\
 				&				   &		 \\ \hline 				
\end{tabular}
\end{center}
\label{MolecularCommandLineArgs}
\end{table}%

\FloatBarrier

Let us consider an example.  Suppose that we would like to execute Ogihara and Ray's algorithm for a DIMACS CNF file.  We would like to execute the instance \texttt{test1.cnf} located in the directory \texttt{/molecularSimulation/testbench}.  We output the results \texttt{test1-o.out} in the same directory as the input CNF.  We invoke Molecular Simulation with the following command.
\[
\texttt{\$ ./execute/simulation i ../testbench/test1.cnf -a o -w ../testbench/test1-o.out}
\]
		
%		<Paragraph> Describe next chapter
In the next chapter, we will describe the automation for a random $k$-{\sc Sat} sweep with each of the algorithms.  The provided Perl scripts are the recommended method for building and execution of Molecular Simulation.

\chapter{Experimental Setup}

%	<Paragraph> Overview of experimental setup

This chapter describes the use of Molecular Simulation for evaluation of a set of DIMACS CNF {\sc Satisfiability} instances.  We discuss configuration for generation of random $k$-{\sc Sat} instances.  Further, any existing DIMACS CNF benchmark may be imported for test.  We provide example configuration options for automating the execution of Molecular Simulation.  The example continues with an analysis of runtime metrics for each test instance.  The next chapter provides the results from the $k$-{\sc Sat} sweep experiment.

	\section{Setup}

%		<Paragraph> Describe architecture

In this section, we describe prerequisites for executing a test bench using Molecular Simulation.  Molecular Simulation requires a 64-bit architecture with a UNIX like system with \texttt{gcc} and \texttt{Perl}.  The target system must meet the minimum requirements.  

Building Molecular Simulation can be performed by invoking the \texttt{Perl} script \texttt{build.pl} from the command line.

\begin{center}
\texttt{\$ perl build.pl}
\end{center}

\noindent This script generates an executable \texttt{simulation} in the directory \texttt{molecularSimulation\symbol{92}execute}.  The next sections describe invocation of Molecular Simulation with desired options.  We begin with the creation and importation of DIMACS CNF datasets.

%\par \noindent  Further porting has been made to execute Molecular Simulation on the RIT CS department's Ubuntu machines.		
		

	\section{Create dataset}

%		<Paragraph> Random $k$-{\sc Sat} instances

	We will create a sweep of random $k$-{\sc Sat} instances to observe {\sc Sat} phase transition.  David Wilson's \texttt{ksat.c} generates random $k$-{\sc Sat} instances in DIMACS CNF format.  The program takes four arguments to create a unique DIMACS CNF instance.  Invocation of the program can be performed using the following command.

\begin{center}

\texttt{./execute/ksat} $k$ $n$ $m$ $s$ \texttt{>} \textit{output}\texttt{.cnf}

\end{center}

This generates \textit{output}\texttt{.cnf} in DIMACS CNF format with $k$ variables per clause $n$ variables, $m$ clauses, and random seed $s$.

%		<Paragraph> Random $k$-{\sc Sat} sweep

We use automated \texttt{Perl} scripts to create a sweep of DIMACS CNF instances.  Setup for a sweep configuration includes specifying a set of ratios.  Invocation of the script generates a set of random $k$-{\sc Sat} instances.  The redirected output gets stored in the target directory with the previous file naming convention.  We use the following command to invoke the construction of a sweep of $k$-{\sc Sat} instances.


\begin{center}

\texttt{\$ perl buildGenerate.pl}

\end{center}



%Random $k$-{\sc Sat} instance  



	\section{Import dataset}

%		<Paragraph> Provide DIMACS {\sc Sat} input  

Datasets of DIMACS CNF input may be provided for batch processing.  This includes random $k$-{\sc Sat} instances generated from the previous section, or importing existing DIMACS CNF instances.   

%		<Paragraph> DIMACS Sat benchmarks

DIMACS CNF benchmarks are available for download from: \url{ftp://dimacs.rutgers.edu/pub/challenge/satisfiability/}.


	\section{Configure test}


%		<Paragraph> Select algorithm
%		<Paragraph> Set output options
%		<Paragraph> Select input CNF file
The previous chapter described a single execution of Molecular Simulation.  Now we provide the automated invocation for processing datasets with each of the algorithms.

The provided \texttt{Perl} script \texttt{executeMolecularSat.pl} allows execution for a directory of DIMACS CNF input.  Executing the script from the command line without arguments processes the experimental setup and saves output to the same directory.

\[
\texttt{ \$ perl executeMolecularSat.pl [options]}
\]

The options for \texttt{executeMolecularSat.pl} can be a combination of the options in Table \ref{executeMolecularSatTable}.




\begin{table}[htdp]
\caption{Command line options for \texttt{executeMolecularSat.pl}}
\begin{center}
\begin{tabular}{|c|c|l|}
\hline
\textbf{Argument} & \textbf{Parameters} & \textbf{Description} \\ \hline
%\texttt{i}		& \texttt{[input]} & Required DIMACS CNF input \\
% 				&				   &		 \\
 \texttt{-d}	& 				   & Distribution algorithm		 \\
 \texttt{-l}	& 				   & Lipton's algorithm		 \\
 \texttt{-o}	& 				   & Ogihara and Ray's algorithm		 \\
 				&				   &		 \\ \hline 				
\texttt{-debug}		&				   & Debug		 \\ 				
 				&				   &		 \\ \hline
\texttt{-p}		&				   & Specify CNF file path. 	 \\
 				& \texttt{[CNF file path]}  &		Default path: \texttt{data/testCNF}	 \\ 			
 				&				   &		\\ \hline 	
\texttt{-f}		&				   & Write output to file		 \\
 				&				   &		\\ \hline 				
\end{tabular}
\end{center}
\label{executeMolecularSatTable}
\end{table}%

\FloatBarrier

%The script automates the execution of every \texttt{*.cnf} file contained in \texttt{[directory]} for each algorithm.

	\section{Execution and collection of data}

%		<Paragraph> Describe understanding data

The output can be analyzed after the automated tests have completed.  The output consists of the standard {\sc Sat} Competition output appended with custom runtime metric logging.  We discuss viewing output directly during execution and reading saved output files.  Collections of output files may be read by the data visualization program and exported into a condensed table. 

		\subsection{Execution output}

%			<Paragraph> Save output to file

	Molecular Simulation, by default, writes output to standard output on the console.  The \texttt{-f} option saves output to a file as \texttt{[filename]-<a>.out}.  The \texttt{[filename]} consists of the DIMACS CNF name and \texttt{<a>} specifies the algorithm type: \texttt{d}, \texttt{l} or \texttt{o}.

%			<Paragraph> View output on terminal
%			<Paragraph> Verbose output
	Output directed to standard output conforms to the {\sc Sat} Competition rules.  This output may be used during testing, or redirected to an external stream.  The debug option \texttt{-debug} provides detailed information about the execution.  The debug option writes verbose content based on the program execution.  

%		<Paragraph> Output metrics
	Reading output metrics from the saved output, as defined in Table \ref{outputTableDefiniton}, allows for analysis of collected data.  The data visualization reads a directory of output and condenses it as a \texttt{*.tsv} file.  Subsequent datapoint browsing and the online view use the \texttt{*.tsv} file for condensed reading and transmission.  In the next chapter, we provide the results of the experimental setup.

\chapter{Results}

%	<Paragraph> Overview of results
This chapter presents results of the $k$-{\sc Sat} execution test from the previous chapter.  We consider the results of the test and analyze the algorithm metrics.  

	\section{Algorithm metric comparison}
	
%		<Paragraph> Summary of measured metrics
This section describes the results from the simulation.  We analyze the molecular operations count for append, extract, mix, purify, splice, and split.  Presentation of actual computation time and required memory for the solution representation allow for comparison of algorithms.

%\subsection{Split}
%%%%%%%%%%%%%%%%%%%%%%%%%%%%%%%%%
\begin{figure}[htdp]

\begin{center}

\includegraphics[width=0.8\textwidth]{./figures/key.pdf}

\caption{Key for output metrics.  Large shapes represent satisfiable instances and small shapes represent unsatisfiable instances.  Datapoints for Lipton's algorithm are represented with red circles, Ogihara and Ray's algorithm with green squares, and the Distribution algorithm with blue triangles. }
\label{metricKey}
\end{center}
\end{figure}
%%%%%%%%%%%%%%%%%%%%%%%%%%%%%%%%%
\FloatBarrier

%\subsection{Append}
%%%%%%%%%%%%%%%%%%%%%%%%%%%%%%%%%
\begin{figure}[htdp]

\reversemarginpar{
\textbf{Append} concatenates two oligonucleotides. \\

The Distribution algorithm is exponential in the number of appends.  The operation count for append depends on the parsing order of the CNF instance.\\

Lipton's and Ogihara-Ray's algorithms use a fixed number of appends.  This depends on the number of variables and clauses present in the CNF instance.
}

\begin{center}

\includegraphics[width=1.1\textwidth]{./figures/metricOutput/Append.pdf}

\caption{Clause to variable ratio $\alpha$ vs. Number of appends }
\label{appendFig}
\end{center}
\end{figure}
%%%%%%%%%%%%%%%%%%%%%%%%%%%%%%%%%

\FloatBarrier
			
%\subsection{Extract}
%%%%%%%%%%%%%%%%%%%%%%%%%%%%%%%%%
\begin{figure}[htdp]

\reversemarginpar{

\textbf{Extract} filters oligonucleotides from a tube.\\

Ogihara-Ray's algorithm requires the greatest number of extracts.  Lipton's algorithm is linear on $\alpha$ and varies a constant from Ogihara-Ray's algorithm.\\

The Distribution algorithm does not require extract.

}

\begin{center}

\includegraphics[width=1.1\textwidth]{./figures/metricOutput/Extract.pdf}

\caption{Clause to variable ratio $\alpha$ vs. Number of extracts }
\label{extractFig}
\end{center}
\end{figure}
%%%%%%%%%%%%%%%%%%%%%%%%%%%%%%%%%
\FloatBarrier			
			
%\subsection{Mix}
%%%%%%%%%%%%%%%%%%%%%%%%%%%%%%%%%
\begin{figure}[htdp]

\reversemarginpar{
\textbf{Mix} combines the contents of two tubes.\\

Lipton's algorithm requires a linear number of mixes on $\alpha$.  The Distribution algorithm also requires a linear number of mixes, varying by a constant factor from Lipton's algorithm.\\

Ogihara-Ray's algorithm requires a constant number of mixes on $\alpha$.
}

\begin{center}

\includegraphics[width=1.1\textwidth]{./figures/metricOutput/Mix.pdf}

\caption{Clause to variable ratio $\alpha$ vs. Number of mixes }
\label{mixFig}
\end{center}
\end{figure}
%%%%%%%%%%%%%%%%%%%%%%%%%%%%%%%%%
\FloatBarrier

%\subsection{Purify}
%%%%%%%%%%%%%%%%%%%%%%%%%%%%%%%%%
\begin{figure}[htdp]

\reversemarginpar{
\textbf{Purify} ensures a uniform distribution of each independent oligonucleotide in a tube.\\

All three algorithms operate using a linear number of purifications on $\alpha$.  Ogihara-Ray's algorithm requires the greatest number of purifications.  The purifications vary by a constant when compared to Lipton's and the Distribution algorithms.

 }

\begin{center}

\includegraphics[width=1.1\textwidth]{./figures/metricOutput/Purify.pdf}

\caption{Clause to variable ratio $\alpha$ vs. Number of purifies }
\label{purifyFig}
\end{center}
\end{figure}
%%%%%%%%%%%%%%%%%%%%%%%%%%%%%%%%%
\FloatBarrier

%\subsection{Splice}
%%%%%%%%%%%%%%%%%%%%%%%%%%%%%%%%%
\begin{figure}[htdp]

\reversemarginpar{
\textbf{Splice} cuts an oligonucleotide at a targeted location.\\

The Distribution algorithm is exponential in the number of splices.  The number of splices depends on the parsing order of the CNF instance.  Each split requires reassembly, accomplished using two appends.  Figure \ref{appendFig} shows the number of appends.\\

Lipton's and Ogihara-Ray's algorithms do not require the splice operator.
}

\begin{center}

\includegraphics[width=1.1\textwidth]{./figures/metricOutput/Splice.pdf}

\caption{Clause to variable ratio $\alpha$ vs. Number of splices }
\label{spliceFig}
\end{center}
\end{figure}
%%%%%%%%%%%%%%%%%%%%%%%%%%%%%%%%%
\FloatBarrier

%\subsection{Split}
%%%%%%%%%%%%%%%%%%%%%%%%%%%%%%%%%
\begin{figure}[htdp]

\reversemarginpar{

\textbf{Split} portions a tube into two exact tubes.\\

The Distribution algorithm requires a linear number of splits.\\

Lipton's and Ogihara-Ray's algorithms are constant in splits based the number of variables.
 }

\begin{center}

\includegraphics[width=1.1\textwidth]{./figures/metricOutput/Split.pdf}

\caption{Clause to variable ratio $\alpha$ vs. Number of splits }
\label{splitFig}
\end{center}
\end{figure}
%%%%%%%%%%%%%%%%%%%%%%%%%%%%%%%%%
\FloatBarrier

			
%\subsection{Time}
%%%%%%%%%%%%%%%%%%%%%%%%%%%%%%%%%
\begin{figure}[htdp]

\reversemarginpar{
\textbf{Time} measures algorithm execution time in seconds.\\

Ogihara-Ray's algorithm requires the least time.  In cases where the {\sc Satisfiability} instance is under-constrained, where more possible solutions occur, the algorithm takes the greatest time.  Less pruning occurs in over-constrained instances, reducing the execution time of test instances.\\

Lipton's algorithm executes in exponential time $\alpha \approx [4.2, 8.2]$ (the phase transition region for 3-{\sc Sat}) taking the longest.\\

The Distribution algorithm executes in exponential time, and performs better than Lipton's algorithm for low conflict ratios.  However over the entire sweep performs worse than both Lipton's and Ogihara-Ray's algorithms.  It shares the same $\alpha \approx [4.2, 8.2]$ during the $3$-{\sc Sat} phase-transition.
}

\begin{center}

\includegraphics[width=1.1\textwidth]{./figures/metricOutput/Time.pdf}

\caption{Clause to variable ratio $\alpha$ vs. execution time in seconds }
\label{timeFig}
\end{center}
\end{figure}
%%%%%%%%%%%%%%%%%%%%%%%%%%%%%%%%%

\FloatBarrier
			
%\subsection{Solution space}
%%%%%%%%%%%%%%%%%%%%%%%%%%%%%%%%%
\begin{figure}[htdp]

\reversemarginpar{
\textbf{Memory} measures witness footprint in Bytes.\\

Lipton's and Ogihara-Ray's algorithms share the same solution footprint.\\

The Distribution algorithm contains a larger solution footprint after the trivially satisfiable instances with $\alpha \approx [0.2, 0.8]$.  The space contains a set of non-conflicting assignments from $\alpha \approx [0.8, 2.9]$.  Non-conflicting assignments consist of witnesses for only necessary literals. \\

Each {\sc Satisfiability} instance has a constrained solution space during the phase-transition region.  All three algorithms share the same footprint.  There are no satisfiable instances in this test with $\alpha > 7.2$. The axis in Figure \ref{memoryFig} scales accordingly.
}

\begin{center}

\includegraphics[width=1.1\textwidth]{./figures/metricOutput/Memory.pdf}

\caption{Clause to variable ratio $\alpha$ vs. satisfiable solution footprint in Bytes }
\label{memoryFig}
\end{center}
\end{figure}
%%%%%%%%%%%%%%%%%%%%%%%%%%%%%%%%%

\FloatBarrier

\input{ch8_Conclusions.tex}

\newpage
\addcontentsline{toc}{chapter}{Bibliography} 
\bibliographystyle{acm}
\bibliography{./refAll}

\appendix

\chapter{Source}

\section{Contributed}

\noindent Download Molecular Simulation:

\begin{itemize}
	\item \url{https://github.com/dncarley/MolecularSimulation}
\end{itemize}

\section{External}

\noindent Download David Wilson's $k$-{\sc Sat} Generator:

\begin{itemize}
	\item \url{http://research.microsoft.com/en-us/um/people/dbwilson/ksat/default.htm}
\end{itemize}


\section{Dependencies for Molecular Simulation}

	\begin{itemize}
		\item GCC, The GNU Compiler Collection (Required)
			\begin{itemize}
				\item  \url{http://gcc.gnu.org/}
			\end{itemize}		
		\item Perl 5 (Required)
				\begin{itemize}
					\item \url{http://www.perl.org/}
				\end{itemize}
		\item Perl Parallel::ForkManager (Optional for running on multiple cores)
			\begin{itemize}
				\item \url{http://search.cpan.org/~dlux/Parallel-ForkManager-0.7.5}
			\end{itemize}				
		\item Doxygen (Optional for generating documentation)
			\begin{itemize}
				\item \url{http://www.stack.nl/~dimitri/doxygen/}
			\end{itemize}				
	\end{itemize}


\chapter{Molecular algorithm trace}

\section{Example {\sc Satisfiability} instance}


\[
 \phi = (x_1 \vee x_2 \vee \neg x_3) \wedge  (x_2 \vee x_3 \vee \neg x_4) \wedge (\neg x_1 \vee \neg x_3 \vee \neg x_4)
\]


\section{Lipton's Algorithm}

\[
T  = \text{{\sc Combinatorial Generate}}(4)
\]

\[
T = 
\]
\begin{verbatim}
	STTTT SFTTT STFTT SFFTT STTFT SFTFT STFFT SFFFT
	STTTF SFTTF STFTF SFFTF STTFF SFTFF STFFF SFFFF
\end{verbatim}
	
Next select Clause 1:

\[
C_1 = (x_1 \vee x_2 \vee \neg x_3)
\]

\begin{verbatim}
	STTTT SFTTT STFTT SFFTT STTFT SFTFT STFFT SFFFT
	STTTF SFTTF STFTF SFFTF STTFF SFTFF STFFF SFFFF
\end{verbatim}
	
	Extract $x_1$:

\begin{verbatim}
		STTTT      STFTT      STTFT      STFFT     
		STTTF      STFTF      STTFF      STFFF     
\end{verbatim}

	Extract $x_2$:
\begin{verbatim}
		STTTT SFTTT           STTFT SFTFT          
		STTTF SFTTF           STTFF SFTFF          
\end{verbatim}

	Extract $\neg x_3$:
\begin{verbatim}	
		                    STTFT SFTFT STFFT SFFFT
		                    STTFF SFTFF STFFF SFFFF
\end{verbatim}		                    

	Mix contents:

\begin{verbatim}	
		STTTT SFTTT STFTT      STTFT SFTFT STFFT SFFFT
		STTTF SFTTF STFTF      STTFF SFTFF STFFF SFFFF
\end{verbatim}		

Next select Clause 2:

\[
C_2 = (x_2 \vee x_3 \vee \neg x_4)
\]

\begin{verbatim}
	STTTT SFTTT STFTT      STTFT SFTFT STFFT SFFFT
	STTTF SFTTF STFTF      STTFF SFTFF STFFF SFFFF
\end{verbatim}		

	Extract $x_2$:
\begin{verbatim}
		STTTT SFTTT           STTFT SFTFT          
		STTTF SFTTF           STTFF SFTFF          
\end{verbatim}	

	Extract $x_3$:
\begin{verbatim}	
		STTTT SFTTT STFTT                         
		STTTF SFTTF STFTF                         
\end{verbatim}	

	Extract $\neg x_4$:
\begin{verbatim}
		                                       
		STTTF SFTTF STFTF      STTFF SFTFF STFFF SFFFF	
\end{verbatim}

	Mix contents:
\begin{verbatim}
		STTTT SFTTT STFTT      STTFT SFTFT          
		STTTF SFTTF STFTF      STTFF SFTFF STFFF SFFFF
\end{verbatim}

Finally, select Clause 3:	
\[
C_3 = (\neg x_1 \vee \neg x_3 \vee x_4)
\]
\begin{verbatim}
	STTTT SFTTT STFTT      STTFT SFTFT          
	STTTF SFTTF STFTF      STTFF SFTFF STFFF SFFFF
\end{verbatim}	
	Extract $\neg x_1$:
\begin{verbatim}
		     SFTTT                SFTFT          
		     SFTTF                SFTFF      SFFFF
\end{verbatim}

	Extract $\neg x_3$:
\begin{verbatim}
		                    STTFT SFTFT          
		                    STTFF SFTFF STFFF SFFFF
\end{verbatim}	

	Extract $x_4$:
\begin{verbatim}
		STTTT SFTTT STFTT      STTFT SFTFT          
		                                       
\end{verbatim}

	Mix contents:
\begin{verbatim}
		STTTT SFTTT STFTT      STTFT SFTFT          
		      SFTTF            STTFF SFTFF STFFF SFFFF
\end{verbatim}





\section{Ogihara and Ray's Algorithm}

Initialize the tube $T$ with initial vector assignments for variables $x_1$ and $x_2$
\[
	T = \{\texttt{STT}, \texttt{STF}, \texttt{SFT}, \texttt{SFF}\}
\]

\noindent Iterate variable $x_3$:
\[
C_1 = (x_1 \vee x_2 \vee \neg x_3)
\]

\noindent $\neg x_3$ matches $v_3$

\begin{align*}
T_{P1} &= \{\texttt{STT}, \texttt{STF}\}\\
T_{N1} &= \{\texttt{SFT}, \texttt{SFF}\}\\
T_{P2} &= \{\texttt{SFT}\}\\
T_P &= \{\texttt{STT}, \texttt{STF}, \texttt{SFT}\}
\end{align*}

\[
C_2 = (x_2 \vee x_3 \vee \neg x_4)
\]
\noindent $x_3$ or $\neg x_3$ does not match $v_3$

\[
C_3 = (\neg x_1 \vee \neg x_3 \vee x_4)
\]
\noindent $x_3$ or $\neg x_3$ does not match $v_3$

\noindent Append

\begin{align*}
T_P &= \{\texttt{STTT}, \texttt{STFT}, \texttt{SFTT}\}\\
T_N &= \{\texttt{STTF}, \texttt{STFF}, \texttt{SFTF}, \texttt{SFFF}\}
\end{align*}

\noindent Mix

\[
T = \{\texttt{STTT}, \texttt{STFT}, \texttt{SFTT}, \texttt{STTF}, \texttt{STFF}, \texttt{SFTF}, \texttt{SFFF}\}
\]


\noindent Iterate variable $x_4$:
\[
C_1 = (x_1 \vee x_2 \vee \neg x_3)
\]

\noindent $x_4$ or $\neg x_4$ does not match $v_3$

\[
C_2 = (x_2 \vee x_3 \vee \neg x_4)
\]
\noindent $\neg x_4$ matchs $v_3$

\begin{align*}
T_{P1} &= \{\texttt{STTT}, \texttt{SFTT}, \texttt{STTF}, \texttt{SFTF}\}\\
T_{N1} &= \{\texttt{STFT}, \texttt{STFF}, \texttt{SFFF}\}\\
T_{P2} &= \{\texttt{STFT}\}\\
T_P &= \{\texttt{STTT}, \texttt{SFTT}, \texttt{STTF}, \texttt{SFTF}, \texttt{STFT}\}
\end{align*}

\[
C_3 = (\neg x_1 \vee \neg x_3 \vee x_4)
\]
\noindent $x_4$ matches $v_3$

\begin{align*}
T_{P1} &= \{\texttt{SFTT}, \texttt{SFTF}, \texttt{SFFF}\}\\
T_{N1} &= \{\texttt{STTT}, \texttt{STFT}, \texttt{STTF}, \texttt{STFF}\}\\
T_{P2} &= \{\texttt{STTF}, \texttt{STFF}\}\\
T_N &= \{\texttt{SFTT}, \texttt{SFTF}, \texttt{SFFF}, \texttt{STTF}, \texttt{STFF}\}
\end{align*}

\noindent Append

\begin{align*}
T_P &= \{\texttt{STTT}, \texttt{STFT}, \texttt{SFTT}\}\\
T_N &= \{\texttt{STTF}, \texttt{STFF}, \texttt{SFTF}, \texttt{SFFF}\}
\end{align*}

\noindent Mix

\[
T = \{\texttt{STTT}, \texttt{STFT}, \texttt{SFTT}, \texttt{STTF}, \texttt{STFF}, \texttt{SFTF}, \texttt{SFFF}\}
\]


\section{Distribution Algorithm}

Initialize the tube $T$ with the variables from the first clause
\[
	T = \{(x_1), (x_2), (\neg x_3)\}
\]	

\noindent Select Clause 2:

\par $T_1 =	\text{{\sc Insert Literal}}(T, x_2)$
\[
		T_1 =  \{(x_1,x_2), (x_2), (x_2,\neg x_3)\}
\]
\par $T_2 =	\text{{\sc Insert Literal}}(T, x_3)$
\[	
		T_2 =  \{(x_1,x_3), (x_2,x_3)\}	
\]
\par $T_3 =	\text{{\sc Insert Literal}}(T, \neg x_4)$
\[	
		T_3 =  \{(x_1,\neg x_4), (x_2,\neg x_4), (\neg x_3,\neg x_4)\}			
\]		
\par $T = \text{mix}(T_1, T_2, T_3)$
\[	
		T = \{(x_1,x_2), (x_2), (x_2,\neg x_3), (x_1, x_3), (x_2,x_3), (x_1,\neg x_4), (x_2,\neg x_4), (\neg x_3,\neg x_4)\}
\]		

\noindent Select Clause 3:
	
\par $T_1 = 	\text{{\sc Insert Literal}}(T, \neg x_1)$
\[
		T_1 = \{(\neg x_1, x_2), (\neg x_1, x_2,\neg x_3), (\neg x_1, x_2, x_3), (\neg x_1, x_2,\neg x_4), (\neg x_1,\neg x_3,\neg x_4)\}
\]	
\par $T_2 =	\text{{\sc Insert Literal}}(T, \neg x_3)$
\[
		T_2 = \{(x_1,x_2,\neg x_3), (x_2,\neg x_3), (x_2,\neg x_3), (x_1,\neg x_3,\neg x_4), (x_2,\neg x_3,\neg x_4), (\neg x_3,\neg x_4)\}
\]
\par $T_2 =	\text{{\sc Insert Literal}}(T, x_4)$
\[
		T_3 = \{(x_1,x_2,x_4), (x_2,x_4), (x_2,\neg x_3,x_4), (x_1,x_3,x_4), (x_2,x_3,x_4)\}
\]
\par $T = \text{mix}(T_1, T_2, T_3)$
\begin{align*}
		T = \{&(\neg x_1,x_2), (\neg x_1,x_2,\neg x_3), (\neg x_1,x_2,x_3), (\neg x_1,x_2,\neg x_4), (\neg x_1,\neg x_3,\neg x_4),\\
			  &(x_1,x_2,\neg x_3), (x_2,\neg x_3), (x_1,\neg x_3,\neg x_4), (x_2,\neg x_3,\neg x_4), (\neg x_3,\neg x_4),\\
			  &(x_1,x_2,x_4), (x_2,x_4), (x_2,\neg x_3,x_4), (x_1,x_3,x_4), (x_2,x_3,x_4)\}
\end{align*}


\end{document}
