\chapter{Existing molecular algorithms for {\sc Satisfiability}}

%<Paragraph> Introduce two molecular algorithms for {\sc Satisfiability}

In this chapter, we introduce two molecular algorithms for {\sc Satisfiability}.  These algorithms construct sets of all witnesses for {\sc Satisfiability} instances.  Lipton's algorithm requires a combinatorial space of all witness candidates to be constructed and then filters out invalid ones.  Ogihara and Ray's algorithm constructs a set of witness candidates throughout execution.  Following the description, we explore the physical implementations of---and simulation frameworks for---these algorithms.

\begin{table}[htdp]
\caption{Components of Boolean literals and equivalent literal representations.}
\begin{center}
\begin{tabular}{| c | c | c | c | c |}
\hline
\textbf{Literal} & \textbf{Variable} $(v)$ & \textbf{Polarity} $(P)$ & \textbf{DIMACS} $(\pm v)$ & \textbf{Condensed} $(v_P)$ \\ \hline	
$x_1$ & $1$ & $\texttt{T}$ & $1$ & $1_{\texttt{T}}$ \\
$\neg x_1$ & $1$ & $\texttt{F}$ & $-1$ & $1_{\texttt{F}}$ \\
$x_n$ & $n$ & $\texttt{T}$ & $n$ & $n_{\texttt{T}}$ \\
$\neg x_n$ & $n$ & $\texttt{F}$ & $-n$ & $n_{\texttt{F}}$ \\ \hline
\end{tabular}
\end{center}
\label{equivalentLiteralTable}
\end{table}%

The algorithm definitions and example traces use the literal conventions listed in Table \ref{equivalentLiteralTable}.  Table \ref{equivalentLiteralTable} lists components of a literal (variable and polarity), along with equivalent forms (DIMACS and a condensed representation).  

In Chapter 1, witness candidates for {\sc Satisfiability} were represented as a bit-vector $B$.  Consider the equivalent representation for witness candidates in Figure \ref{equivalentWitnessRepresentations}.

\begin{figure}[htbp]
\begin{center}

	\begin{align*}
	B &= [0, 1, 0, 1 \rangle \\
	L &= \{ \neg x_1, x_2, \neg x_3, x_4 \} \\
	V &= \{ 1_{\texttt{F}}, 2_{\texttt{T}}, 3_{\texttt{F}}, 4_{\texttt{T}} \} \\
	D &= \texttt{SFTFT} 
	\end{align*}

\caption{The bit-vector $B = [0, 1, 0, 1 \rangle$ can be represented as the set of literal assignments $L = \{ \neg x_1, x_2, \neg x_3, x_4 \}$.  Using the condensed representation from Table \ref{equivalentLiteralTable} we can represent the set in condensed form $V = \{ 1_{\texttt{F}}, 2_{\texttt{T}}, 3_{\texttt{F}}, 4_{\texttt{T}} \}$.  A directed polarity string with initial sequence (\texttt{S}), followed by a sequence of literals provides $D = \texttt{SFTFT}$.}
\label{equivalentWitnessRepresentations}
\end{center}
\end{figure}

\FloatBarrier

We use the directed polarity string (representation $D$ in Figure \ref{equivalentWitnessRepresentations}) as shorthand for a directed oligonucleotide.  The directed string representation $D$ can be indexed by the variable $v$.  In Lipton's algorithm, literal configurations for a variable $v$ get extracted directly ($v_{\texttt{T}}$ or $v_{\texttt{F}}$).

In Ogihara and Ray's algorithm, we extract satisfying literal configurations from an ordered clause $[a, b, c]$.  We use the literal configuration $v_P$ to extract the satisfying literal $P$ (either \texttt{T} or \texttt{F}) for the variable $v$.  We use the literal configuration $v_N$ to extract any other non-satisfiable literal assignments to $v$.

\section{Lipton's algorithm for {\sc Satisfiability}}

%	<Paragraph> Introduce Lipton's algorithm

Introduced in 1995 by Richard Lipton \cite{Lipton95usingdna}, this algorithm filters satisfiable witnesses from a combinatorial space of all witness candidates.  Lipton's algorithm is analogous to a conventional brute-force search for all witnesses of a {\sc Satisfiability} instance.

{\sc Lipton's Algorithm} (Algorithm 3.1.1) first constructs a combinatorial space $T$ containing oligonucleotide configurations for all potential witness candidates.  The algorithm iterates over each of the clauses $C$ in $\phi$.  

From each clause $C$, each of the literals contained within each clause $C$ get filtered to satisfiable witnesses.  The contents of $T$ get filtered by incrementally extracting those literals that satisfy the current clause.  The next iteration filters witnesses that satisfy the previous clauses from $T$ and the literal contents from $C$.  The algorithm terminates with a set of witnesses $T$ for the CNF input $\phi$.  If $\phi$ is unsatisfiable, then $T = \emptyset$.  

% We iterate over each of the clauses in $\phi$.  Each clause iteration filters the contents of the tube $T$ to only witnesses that satisfy the current clause.  

	
%%%%%%%%%%%%%%%%%%%%%%%
%%%%%%%%%%%%%%%%%%%%%%%
\begin{pseudocode}{Lipton's Algorithm}{\phi}
n \text{ number of variables in } \phi \\
\\
T \GETS \text{{\sc Combinatorial Generate}}(n) \\
\FOREACH \text{clause } C \text{ in } \phi \DO
	\BEGIN
	T_c \GETS \emptyset \\
	\FOREACH \text{variable } v \text{ in } C \DO
		\BEGIN
			\IF v \text{ is a positive literal} \THEN
				\BEGIN
					T_P \GETS \text{extract}(T, +v)\\
					T_c \GETS \text{mix}(T_P, T_c)						
				\END
			\ELSE
				\BEGIN
					T_N \GETS \text{extract}(T, -v)\\
					T_c \GETS \text{mix}(T_N, T_c)						
				\END
		\END
	\\
	T \GETS T_c\\
	\END
\\
\RETURN{\text{detect}(T)}
\end{pseudocode}

\FloatBarrier

	\subsection{Description of Lipton's algorithm}
		
%		<Paragraph> Describe setup
The function {\sc Combinatorial Generate} (Algorithm 3.1.2) implements the split-mix synthesis technique \cite{furka1982, furkaBook}.  {\sc Combinatorial Generate} returns a tube $T_{comb}$ consisting of oligonucleotides that represent all $2^n$ distinct witness candidates.  The tube $T_{comb}$ begins with an initial medium denoted by \texttt{S}.  An iterative loop extends $T_{comb}$ using split-mix synthesis.  Each split corresponds with appending the tubes with a true (\texttt{T}) and false (\texttt{F}) assignment.  The two tubes are mixed and purified to contain equimolar portions of each witness candidate.  


\begin{figure}[htbp]
	\renewcommand{\figurename}{Algorithm}
	\renewcommand{\thepseudocode}{\ref{combinatorialGenerate}}
	
	\begin{center}

	\begin{pseudocode}[shadowbox]{Combinatorial Generate}{n}
		T_{comb} \GETS \{\texttt{S}\} \\
	
		\FOR v \GETS 1 \text{ to } n \DO
			\BEGIN
			
			[T_1,T_2] \GETS \text{split(} T_{comb}\text{)}\\
			T_1 \GETS \text{append(}T_1, v_{\texttt{T}} \text{)}\\
			T_2 \GETS \text{append(}T_2, v_{\texttt{F}} \text{)}\\
			T_{comb} \GETS \text{mix(}T_1,T_2\text{)}\\
		\END
		\\
		T_{comb} \GETS \text{purify(}T_{comb}\text{)} \\
		\RETURN {T_{comb}}
	\end{pseudocode}

\caption{{\sc Combinatorial Generate} constructs a combinatorial space consisting of $2^n$ molecular configurations in polynomial time.}
\label{combinatorialGenerate}
\end{center}
\end{figure}

\FloatBarrier

Let us consider an example execution of {\sc Combinatorial Generate} with $n = 2$.

\noindent The tube $T_{comb}$ begins as an empty tube.  A start configuration \texttt{S} initiates the tube $T_{comb}$ with a medium for combinatorial synthesis.

\noindent We begin with the initial contents
\[
T_{comb} = \{ \texttt{S} \}.
\]

\noindent Iteration $v = 1$:

First, split the contents of $T_{comb}$.  We have
\begin{align*}
 T_1 &= \{ \texttt{S} \}, \text{ and} \\
 T_2 &= \{ \texttt{S} \}.
\end{align*}

Next, append each of the tubes with a positive (\texttt{T}) and negative (\texttt{F}) assignment for the literal $v_1$.  We have
\begin{align*}
 T_1 &= \{ \texttt{ST} \}, \text{ and} \\
 T_2 &= \{ \texttt{SF} \}.
\end{align*}

Mix the contents of $T_1$ and $T_2$ to form $T_{comb}$ for the next iteration.  We have
\[
T_{comb} = \{\texttt{ST}, \texttt{SF}\}.
\]

\noindent Iteration $v = 2$:

Split the contents of $T_{comb}$.  We have
\begin{align*}
 T_1 &= \{\texttt{ST}, \texttt{SF}\}, \text{ and} \\
 T_2 &= \{\texttt{ST}, \texttt{SF}\}.
\end{align*}

Next, append each of the tubes with a positive (\texttt{T}) and negative (\texttt{F}) assignment for the literal $v_2$.  We have
\begin{align*}
 T_1 &= \{\texttt{STT}, \texttt{SFT}\}, \text{ and} \\
 T_2 &= \{\texttt{STF}, \texttt{SFF}\}.
\end{align*}

Mix the contents of $T_1$ and $T_2$ to form $T_{comb}$ for the final iteration.  The algorithm {\sc Combinatorial Generate} returns the following tube
\[
T_{comb} = \{\texttt{STT}, \texttt{SFT}, \texttt{STF}, \texttt{SFF}\}.
\]

%%%%%%%%%%%%%%%%%

{\sc Combinatorial Generate} generates all witness candidates for a {\sc Satisfiability} instance.  {\sc Lipton's Algorithm} filters, from a combinatorial space $T$, configurations that represent witnesses for the input $\phi$.

%		<Paragraph> Describe execution
%The amplification process gets modeled with a purification step.  This eliminates all redundant strings for the simulated implementation.  After the iteration completes, the complete combinatorial space gets returned. This space consists of $2^n$ vectors of length $n$. 


%		<Paragraph> Describe Output
	\subsection{Detailed trace of Lipton's algorithm}
	
%	<Paragraph> Introduce pseudocode
Appendix B lists a detailed execution trace for Lipton's algorithm.

%%%%%%%%%%%%%%%%%%%%%%%%%%%%%%%%%

\section{Ogihara and Ray's algorithm for {\sc Satisfiability}}

%	<Paragraph> Introduce Ogihara and Ray's algorithm

Ogihara and Ray's algorithm (Algorithm 3.2.1) consist of a breadth-first evaluation of clauses from a CNF formula \cite{Ogihara:1996:BFS:898228,Ogihara97dna-basedparallel}.  The algorithm constructs a set of witness candidates based on a parse of a 3-CNF formula.  In this section, we describe the preconditions and execution of Ogihara and Ray's algorithm.

%		<Paragraph> Describe execution

\begin{figure}[htbp]
	\renewcommand{\figurename}{Algorithm}
	\renewcommand{\thepseudocode}{\ref{ogiharaRayAlgorithm}}
	
	\begin{center}

	\begin{pseudocode}[shadowbox]{Ogihara and Ray's Algorithm}{\phi}
	
	\text{// Input $\phi$ consists of $n$ variables.}\\
	\text{// Each clause $C$ contains ordered literals $(a,b,c)$.}\\
	\text{// Extract literals using the condensed notation $v_P$, where: }\\
	\text{// \hspace{1em} $v_P$ matches the literal, and $v_N$ matches the negated literal.}\\
	\\
	T \GETS \{ \texttt{STT}, \texttt{STF}, \texttt{SFT},  \texttt{SFF}\} \\
	
	\FOR v \GETS 3 \text{ to } n \DO
		\BEGIN
		[T_P, T_N] \GETS \text{split}(T)\\
	
		\FOREACH \text{clause } C \text{ in } \phi \DO
			\BEGIN
				(a, b, c) \GETS C\\
				\IF v_{\texttt{T}} = c  \THEN
					\BEGIN
						T_{P1} \GETS \text{extract}(T_N, a_P)\\
						T_{N1} \GETS \text{extract}(T_N, a_N)\\				
						T_{P2} \GETS \text{extract}(T_{N1}, b_P)\\
						T_{N} \GETS \text{mix}(T_{P1}, T_{P2})\\
						T_{N} \GETS \text{purify(}T_{N}\text{)}					
					\END \\  
				\IF v_{\texttt{F}} = c \THEN
					\BEGIN
						T_{P1} \GETS \text{extract}(T_P, a_P)\\
						T_{N1} \GETS \text{extract}(T_P, a_N)\\				
						T_{P2} \GETS \text{extract}(T_{N1}, b_P)\\
						T_{P} \GETS \text{mix}(T_{P1}, T_{P2})\\
						T_{P} \GETS \text{purify(}T_{P}\text{)} 						
					\END\\
			\END\\
			T_P \GETS \text{append}(T_P, v_{\texttt{T}})\\
			T_N \GETS \text{append}(T_N, v_{\texttt{F}})\\
			T \GETS \text{mix}(T_P, T_N)\\
			T \GETS \text{purify(}T\text{)} \\									
		\END\\
	\RETURN{\text{detect}(T)}
	\end{pseudocode}

\caption{{\sc Ogihara and Ray's Algorithm} evaluates each subsequent variable and determines possible assignments.  The possible assignments for the variables $a$ and $b$ get extracted if $c$ matches the current variable $v$.  Effectively pruning only potential solutions.  These potential solutions $T_P$ and $T_N$ get appended with the positive or negative string assignments.  The algorithm continues until each variable gets evaluated.  The remaining space $T$ contains all solutions for the CNF instance $\phi$ after the algorithm terminates.}
\label{ogiharaRayAlgorithm}
\end{center}
\end{figure}

\FloatBarrier

\subsection{Description of Ogihara and Ray's algorithm}
		
%		<Paragraph> Describe preconditions
Ogihara and Ray's algorithm requires each input to have the two attributes.

\begin{enumerate}
\item All clauses consist of exactly three literals
\item All clauses must be sorted by the literal's variable
\end{enumerate}

Considering only $3$-{\sc Sat} instances fulfills Attribute (1).  If models of $k$-{\sc Sat} with $k > 3$ are desired as input, then a polynomial time reduction to $3$-{\sc Sat} must occur prior to execution.

Sorting the literals from each clause by variable fulfills Attribute (2).  The polarity of each literal may consist of a positive or negative assignment.  Construction of witness candidates consider the possibilities of the clause ordering

\[
a < b < c.
\]

%		<Paragraph> Describe setup

{\sc Ogihara and Ray's Algorithm} begins with four initial witness candidates. 
\[
T = \{ \texttt{STT}, \texttt{STF}, \texttt{SFT}, \texttt{SFF}\}
\]

\noindent For example, let us evaluate the clause
\[
C = x_1 \vee \neg x_2 \vee \neg x_3.
\]

\noindent On the first iteration, we compare the third ordered literal $c$ with $x_3$.  Since $c = \neg x_3$, extract configurations that satisfy $a \vee b$.  

\begin{align*}
T_P &= \{ \texttt{STT}, \texttt{STF}, \texttt{SFT}, \texttt{SFF}\} \\
T_N &= \{ \texttt{STT}, \texttt{STF}, \texttt{SFT}, \texttt{SFF}\} 
\end{align*}

\noindent From $T$, select configurations that satisfy $a_P = a_{\texttt{T}}$
\[
T_{P1} = \{ \texttt{STT}, \texttt{STF} \}.
\]

\noindent From $T$, Select configurations that satisfy $a_N = a_{\texttt{F}}$
\[
T_{N1} = \{ \texttt{SFT}, \texttt{SFF} \}.
\]

\noindent From $T_{N1}$, select configurations that satisfy $b_P = b_{\texttt{F}}$
\[
T_{P2} = \{ \texttt{SFF} \}.
\]

\noindent Mix the contents of $T_{P1}$ and $T_{P2}$ as the contents of $T_P$

\[
T_P = \{ \texttt{STT}, \texttt{STF}, \texttt{SFF} \}.
\]

\noindent We have the tubes

\begin{align*}
T_P &= \{ \texttt{STT}, \texttt{STF}, \texttt{SFF} \}, \\
T_N &= \{ \texttt{STT}, \texttt{STF}, \texttt{SFT}, \texttt{SFF}\}.
\end{align*}

\noindent Finally append assignments that satisfy the current literal with $T_P$ and $T_F$.

\begin{align*}
T_P &= \text{append}(T_P, c_{\texttt{F}}) \\
	&= \{ \texttt{STTF}, \texttt{STFF}, \texttt{SFFF} \}
\end{align*}

\begin{align*}
T_N &= \text{append}(T_N, c_{\texttt{T}}) \\
	&= \{ \texttt{STTT}, \texttt{STFT}, \texttt{SFTT}, \texttt{SFFT}\}
\end{align*}

\noindent Mix the contents of $T_P$ and $T_N$ to form the set of configurations that witness the clause. 
\[
T = \{ \texttt{STTF}, \texttt{STFF}, \texttt{SFFF},  \texttt{STTT}, \texttt{STFT}, \texttt{SFTT}, \texttt{SFFT}\}
\]
		
	\subsection{Detailed trace of Ogihara and Ray's algorithm}
	
Appendix B lists a detailed execution trace for Ogihara and Ray's algorithm.

\section{Implementations of molecular {\sc Satisfiability} solvers}

We introduce existing implementations of molecular {\sc Satisfiability} solvers.  Physical implementations apply molecular biology techniques using real molecules.  Simulation frameworks use standard computation to simulate molecular biology techniques.

	\subsection{Physical implementations}
	
%	<Paragraph> Describe laboratory 

Yoshida and Suyama implemented Ogihara and Ray's algorithm using manual molecular biology techniques \cite{dnaBasedImplemetation_Yoshida2000}.  This experiment solved a 3-CNF instance with four variables and 10 clauses.

Braich et al. implemented a molecular computer to filter solutions for a 3-{\sc Sat} instance \cite{Braich02solutionof}.  This experiment solved a 3-CNF instance with 20 variables and 24 clauses.
	
	\subsection{Simulation frameworks}

%		<Paragraph> Describe computer simulation 
Martín-Mateos et al. introduced a simulation for Lipton's algorithm \cite{MartinMateos02molecularcomputation}.   Molecular operations get implemented in \texttt{ACL2}, a Common Lisp variant.  The framework for this system implemented test cases for Lipton's algorithm.

Ogihara provides test results for implementation of his original molecular algorithm \cite{Ogihara:1996:BFS:898228}.  This simulation provides a comparison to Lipton's algorithm for practical length restrictions.
