\chapter{Conclusions}

%	<Paragraph> Summary of project
This project considered {\sc Satisfiability} as a problem for general computation.  We considered three molecular algorithms for {\sc Satisfiability} and simulated their execution with a conventional computing implementation.  In this chapter, we state the contributions of this project and directions molecular computation will take.
	
	\section{Contributions}

%		<Paragraph> Molecular algorithm for {\sc Sat}
%		<Paragraph> Simulation of molecular {\sc Sat} algorithms
We developed several contributions for molecular computing during this project.  This includes introducing the molecular Distribution algorithm for {\sc Satisfiability} in Chapter 4.  We introduced Molecular Simulation in Chapter 5 and collected data from simulations of three molecular {\sc Satisfiability} algorithms described in Chapter 6.  

%		<Paragraph> Generalized computing architecture 		

	\section{Future work}
	
%		<Paragraph> Implementation of molecular computers
Nanopore sequencers have been designed for reading molecules and diagnosing patients in a medical setting.  Creation of molecular computation architectures permit generalized computation with physical encodings.  

Sequencing for medical purposes will continue to drive innovation. Construction of the molecular architechure provides tools beyond general purpose computing. The computing architecture operates as a physical molecular laboratory.  This provides the ability to replicate molecular interactions in a controlled setting and observe real interactions of genetics.
%		<Paragraph> Application of Molecular computing as a gene laboratory 
		
