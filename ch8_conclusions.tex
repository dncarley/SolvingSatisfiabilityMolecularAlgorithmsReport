\chapter{Conclusion}

%	<Paragraph> Summary of project

We considered three molecular algorithms for {\sc Satisfiability} and simulated their execution with a computation framework.  Chapter 2 discussed techniques for molecular computation and defined a molecular instruction set (Adleman's molecular toolbox).

In Chapter 2, we defined {\sc Satisfiability} along with the standard forms: CNF, $k$-CNF, and $k$-{\sc Sat}.  We use random $3$-CNF instances for execution of simulated molecular algorithms.  
	
	\section{Contribution}

This project introduces the Distribution algorithm for {\sc Satisfiability}.  This algorithm distributes the literals into a growing set of witnesses during the evaluation of a $k$-CNF instance.

We provide comparisons of molecular operators for each of the molecular algorithms under test.  The software framework for this project (Molecular Simulation) was used for simulating and collecting metrics on random $3$-CNF instances at fixed $n=20$ spanning clause-variable ratios $\alpha = m/n = [0.2, 14.0]$ in increments of $0.2$.

	\section{Future work}
	
%		<Paragraph> Implementation of molecular computers

Variations on the Distribution algorithm can decrease the length and quantity of witness candidates.  In particular, the Distribution algorithm may select clauses with the maximum number of idempotent literal assignments while regarding all other literals as unassigned.  From the remaining clauses, assignments can be made using the standard Distribution algorithm (Chapter 4) or employ a clause selection strategy.\\

In this project we considered the use of a genetic substrate to store and match witness candidates for {\sc Satisfiability}.  Existing implementation of molecular computers use laboratory procedures operated by a human or human-like automation.  Gene sequencers currently provide an integrated means for reading natural gene expressions.  

Practical molecular computation must be integrated into a molecular computing architecture.  This architecture must facilite the storage and transfer of isolated tubes with read/write/extract capabilities.  This type of machine provides an environment for observing natural interactions of genes (along with sequencing), and an architecture for molecular computation. 


