\chapter{Conclusions}

%	<Paragraph> Summary of project
This project considered {\sc Satisfiability} as a problem for general computation.  We considered three molecular algorithms for {\sc Satisfiability} and simulated their execution with a computation framework.  In this chapter, we state the contributions of this project and directions molecular computation will take.
	
	\section{Contributions}

%		<Paragraph> Molecular algorithm for {\sc Sat}
%		<Paragraph> Simulation of molecular {\sc Sat} algorithms
We developed several contributions for molecular computation during this project.  This includes introducing the Distribution algorithm for {\sc Satisfiability} in Chapter 4.  We introduced Molecular Simulation in Chapter 5 and collected data from simulations of three molecular {\sc Satisfiability} algorithms described in Chapter 6.  This comparison shows advantages and disadvantages of each of the molecular algorithms for {\sc Satisfiability}.

%		<Paragraph> Generalized computing architecture 		

	\section{Future work}
	
%		<Paragraph> Implementation of molecular computers
Gene sequencers have been designed for reading molecules and diagnosing patients in a medical setting.  These gene sequencers currently have the ability to sequence any type of gene.  Observing real interactions between sequences in a controlled environment permit molecular computation for targeted gene disease diagnosis.

{\sc Satisfiability} defines a canonical input format for combinatorial problems.  Applications of molecular {\sc Satisfiability} algorithms may extend to witness natural gene expressions.  Gene sequencers designed for molecular computation permit observation of molecular interactions on both synthetic and natural gene expressions. 
